\section{Categories, functors, natural transformations}\label{categories}
Office hours. Hood Chatham's are Mondays, 1:30 - 3:30, in 2-390A. Miller's is Tuesdays, 3:00-5:00 in 2-478.

The collection of all topological spaces forms a category. Similarly with simplicial sets, denoted $\set_\Deltab$ or $s\set$. You can take $X\mapsto \Sin_\bullet(X)$, and if $X\to Y$ is a map of spaces, this induces $\Sin_\bullet(X)\to\Sin_\bullet(Y)$. This is a functor $\mathbf{Top}\to s\set$.
\begin{definition}
A category $\cc$ is a class $\mathrm{ob}(\cc)$ of objects, such that for every two objects, $\cc(X,Y)$ is a set (thought of as a set of maps from $X$ to $Y$); these are called morphisms. These satisfy the properties that for all $X\in\mathrm{ob}(\cc)$, there exists an element $1_X\in\cc(X,X)$ that acts as the identity, and for all $X,Y,X\in\mathrm{ob}(\cc)$, there is a composition $\cc(X,Y)\times\cc(Y,Z)\to\cc(X,Z)$. These in turn satisfy the following properties:
\begin{itemize}
\item $1_Y\circ f=f$, and $f\circ 1_X=f$.
\item Composition is associative.
\end{itemize}
\end{definition}
\begin{example}
The category of sets and functions between sets forms a category. The category of abelian groups and homomorphisms forms a category. Of course, topological spaces and continuous maps. A monoid, which is a category with one object. The simplex category (note that $|[n]|=n+1$). A poset forms a category; but note that if $x\leq y$ and $y\leq x$, then $x\cong y$, not $x=y$. The latter property holds if the only isomorphisms are identities.
\end{example}
A small category is one such that $\mathrm{ob}(\cc)$ is a set, not necessarily a class.
\begin{definition}
Let $\cc,\cd$ be categories. A functor $F:\cc\to\cd$ is a function $\mathrm{ob}(\cc)\to\mathrm{ob}(\cd)$, such that for all $x,y\in\mathrm{ob}(\cc)$, there is a map $\cc(x,y)\to\cc(F(x),F(y))$ that respects composition and the identity.

Let $F,G:\cc\to\cd$. A natural transformation $\theta:F\to G$ is a map $\theta(X):F(X)\to G(X)$ for all $X\in\mathrm{ob}(\cc)$, and a commuting diagram for all $f:X\to Y$:
\begin{equation*}
\xymatrix{F(X)\ar[d]^{F(f)}\ar[r]^{\theta(X)} & G(X)\ar[d]^{G(f)}\\
F(Y)\ar[r]^{\theta(Y)} & G(Y)}
\end{equation*}
\end{definition}
\begin{example}
The boundary map $\partial:S_n\to S_{n-1}$ is a natural transformation. Let $G$ be a group viewed as a one-point category. Any element $F\in\mathrm{Fun}(G,\mathbf{Ab})$ is simply a group action of $G$ on $F(\ast)=A$, i.e., a representation of $G$ in abelian groups. A natural transformation from $F\to F^\prime$ is just a $G$-equivariant map.
\end{example}
