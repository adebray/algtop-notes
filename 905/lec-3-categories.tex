\section{Categories, functors, natural transformations}\label{categories}
%Office hours. Hood Chatham's are Mondays, 1:30 - 3:30, in 2-390A. Miller's is Tuesdays, 3:00-5:00 in 2-478. (commented out because not relevant to readers)
%Replaced \mathbf{C} with \cC as brought up in ``Stuff to fix.''
%Rearranged some stuff. Added a notational remark following the definition of a category.
From spaces and continuous maps, we constructed graded abelian groups and homomorphisms. We now cast this construction in the more general language of category theory.

Our discussion of category theory will be interspersed throughout the text, introducing new concepts as they are needed. Here we begin by introducing the basic definitions.

\begin{definition}
A \emph{category} $\cc$ is a class $\mathrm{ob}(\cc)$ of objects, such that for every two objects $X$ and $Y$ there is a set of \emph{morphisms} $\cc(X,Y)$ (thought of as a set of maps from $X$ to $Y$). We require that for all $X\in\mathrm{ob}(\cc)$, there exists an identity element $1_X\in\cc(X,X)$, and for all $X,Y,Z\in\mathrm{ob}(\cc)$, there is a composition $\cc(X,Y)\times\cc(Y,Z)\to\cc(X,Z)$ sending $(f,g)\mapsto g \circ f$. These in turn satisfy the following:
\begin{itemize}
\item $1_Y\circ f=f$, and $f\circ 1_X=f$.
\item Composition is associative.
\end{itemize}
\end{definition}
Note that, for set-theoretic reasons, we require the collection of objects to be a class. This enables us to talk about a ``category of all sets'' for example, but not a ``category of all categories'' because that is too large.

We will often write $X\in\cc$ to mean $X\in\mathrm{ob}(\cc)$, and $f\colon X\to Y$ to mean $f\in \cc(X,Y)$.
\begin{definition}
If $X,Y\in \cc$, then $f\colon X\to Y$ is an \emph{isomorphism} if there exists $g\colon Y\to X$ with $f \circ g=1_Y$ and $g\circ f=1_X$, and we write $X\cong Y$. 
\end{definition}

\begin{example}
Many common mathematical structures can be arranged in categories.
\begin{itemize}
\item Sets and functions between them form a category $\set$.
\item Abelian groups and homomorphisms form a category $\mathbf{Ab}$.
\item Topological spaces and continuous maps form a category $\mathbf{Top}$.
\item Simplicial sets and their maps form a category $s\set$.
\item A monoid is the same as a category with one object, where the elements of the monoid are the morphisms in the category.
\item The sets $[n]=\{0,\ldots,n\}$ for $n\geq 0$ together with weakly order-preserving maps between them form the simplex category $\Delta$.
\item A poset forms a category in which there is a morphism from $x$ to $y$ iff $x\leq y$. However, note that $x\leq y$ and $y\leq x$ imply $x\cong y$ rather than $x=y$. The latter holds if the only isomorphisms are identities.
\end{itemize}
\end{example}
A small category is one such that $\mathrm{ob}(\cc)$ is a set, not necessarily a class. While we cannot consider the category of all categories, it is sensible to define the category $\mathbf{Cat}$ of all small categories. But first, we must first define a ``morphism of categories.''
\begin{definition}
Let $\cc,\cd$ be categories. A \emph{functor} $F\colon\cc\to\cd$ is a function $\mathrm{ob}(\cc)\to\mathrm{ob}(\cd)$, such that for all $x,y\in\mathrm{ob}(\cc)$, there is a map $\cc(x,y)\to\cc(F(x),F(y))$ that respects composition and the identity.
\end{definition}
The diagram at the beginning of the previous section shows the functors we have built thus far (although explicit verification of functoriality is left to the reader).

We go a step further. Suppose we fix categories $\cc$ and $\cd$ and consider all functors $\cc\to\cd$. What is a ``morphism of functors''?
\begin{definition}
Let $F,G\colon \cc\to\cd$. A \emph{natural transformation} $\theta\colon F\to G$ consists of maps $\theta(X)\colon F(X)\to G(X)$ for all $X\in\mathrm{ob}(\cc)$ such that the following diagram commutes for all $f\colon X\to Y$:
\begin{equation*}
\xymatrix{F(X)\ar[d]^{F(f)}\ar[r]^{\theta(X)} & G(X)\ar[d]^{G(f)}\\
F(Y)\ar[r]^{\theta(Y)} & G(Y)}
\end{equation*}
\end{definition}
Going further down the rabbit hole leads to higher category theory, which we will not delve into.

Natural transformations are central to algebraic topology. We will frequently describe certain maps as ``natural,'' which is to say that they are natural transformations. The reader should determine the functors involved if they are not explicitly stated.
\begin{example}
The boundary map $\partial\colon S_n\to S_{n-1}$ is a natural transformation.

Let $G$ be a group viewed as a one-point category. Any element $F\in\mathrm{Fun}(G,\mathbf{Ab})$ is simply a group action of $G$ on $F(\ast)=A$, i.e., a representation of $G$ in abelian groups. Given another $F^\prime\in\mathrm{Fun}(G,\mathbf{Ab})$ with $F^\prime(\ast)=A^\prime$, then a natural transformation from $F\to F^\prime$ is precisely a $G$-equivariant map $A\to A^\prime$.
\end{example}
