\section{Surfaces and symmetric nondegenerate bilinear forms}
Never mind, I decided to TeX this.
\todo[inline]{I think that this should be in an appendix...}
\subsection{Case of Poincar\'{e} duality}
Let $M$ be a compact manifold of dimension $n$ (this is the kind of thing most of the world cares about). This whole lecture will be happening with coefficients in $\FF_2$. There is a cup product pairing $ H^p(M)\otimes H^q(M)\to H^{p+q}(M)$.
\begin{theorem}
There exists a unique class $[M]\in H_n(M)$, called the fundamental class, such that there's a pairing $ H^p(M)\otimes H^q(M)\xrightarrow{\cup} H^n(M)\xleftarrow{\langle -,[M]\rangle}\FF_2$ when $p+q=n$. This pairing is a nonsingular (aka perfect) pairing, i.e., the map $ H^p(M)\to\Hom( H^q(M),\FF_2)$ defined by $x\mapsto(y\mapsto\langle x\cup y,[M]\rangle)$ is an isomorphism.
\end{theorem}
One thing I should say is that $ H_\ast(M)$ is zero in dimensions above $n$, and $ H_i(M)$ are finitely generated.
\begin{corollary}
From the UCT, it follows that $ H^p(M)\xrightarrow{\cong}\Hom( H^q(M),\FF_2)= H_q(M)$. The corresponding classes under this association are said to be Poincar\'{e} dual to each other.
\end{corollary}
\begin{remark}
This means I get a pairing $ H_q(M)\otimes H_q(M)\to \FF_2$ in homology when $p+q=n$. This is called the \emph{intersection pairing}. If you have one $p$-cycle and one $q$-cycle, this means that I can make them intersect transversely, i.e., they intersect in points -- or not at all. I count the number of points modulo $2$, and this is the element in $\FF_2$.
\end{remark}
\begin{example}
Let $M=T^2=S^1\times S^1$. We know that $ H^1(M)=\FF_2\times\FF_2=\langle x,y\rangle$ where $x^2=0$ and $y^2=0$, and $xy\neq 0$ (this is the generator of $ H^2(M)$). In terms of homology, this means that I can pick cycles on the torus so that they intersect on one point. This makes sense. Also, $x^2=0=y^2$ because the second copy of $x$ and $y$ can just be moved so that they don't intersect.
\end{example}
\begin{example}[Especially interesting case]
If $n=2k$, I get a symmetric bilinear form on $ H_k(M)$ via $ H_k(M)\otimes H_k(M)\to \FF_2$ that is nondegenerate.
\end{example}
\subsection{Symmetric bilinear forms}
I've really screwed up when TeXing this.

If $W\subseteq V$ is a subspace, then a bilinear form on $V$ gives a bilinear form on $W$, but this need not be nonsingular even if the bilinear form on $V$ is. Observe though that the induced bilinear form on $W$ is nonsingular if and only if $W\cap W^\perp=0$. This result should be obvious, but I don't want to say why.

I have a map $V\xrightarrow{\cong} V^\ast$ by sending $v\mapsto(v^\prime\mapsto v\cdot v^\prime)$. Inside $V$, I have $W^\perp$, and I have a quotient $V^\ast\xrightarrow{\Res} W^\ast$, with kernel $\ker\Res$. I want to claim that $V/W^\perp\cong W^\ast$. We know that $\dim W+\dim W^\perp=\dim V=:n$. If $W$ is nonsingular, then $V=W\oplus W^\perp$. This is true as vector spaces, but also as quadratic forms.

To see this, pick a basis $e_1,\cdots,e_k$ for $W$. The quadratic form is represented by some matrix $\begin{pmatrix} e_i\cdot e_j \end{pmatrix}$. It's nonsingular exactly when its determinant is nonzero, i.e., $1$. Then I can also pick a basis for $W^\perp$. Then I have nondegenerate quadratic form on $V$ given by $\begin{pmatrix}\begin{pmatrix} e_i\cdot e_j \end{pmatrix} & 0 \\ 0 & \begin{pmatrix} ... \end{pmatrix}\end{pmatrix}$ where the bottom left thing is that of $W^\perp$. Since the quadratic form on $V$ has determinant $1$, the matrix for $W^\perp$ also has determinant $1$, and hence $V=W\oplus W^\perp$ as quadratic forms.

Now:
\begin{enumerate}
\item Suppose $v\in V$ has $v\cdot v=1$. Then $V=\langle v\rangle\oplus\langle v\rangle^\perp$. This is an orthogonal splitting into nonsingular forms.
\item Assume $v\cdot v=0$ for all\footnote{Your intuition about dot products isn't very useful here, for example the torus. This is because the matrix for the quadratic form on the torus is $\begin{pmatrix}0 & 1 \\ 1 & 0\end{pmatrix}$. Also, $(x+y)\cdot(x+y)=0$. This means that every cycle on the torus can be moved to have no self-intersection, at least modulo $2$.} $v\in V$. Anyway, in our general situation, $v$ might be zero -- so pick $v\neq 0$. But the form is nonsingular, so there exists $w$ such that $v\cdot w\neq 0$, i.e., $v\cdot w=1$. Then $\langle v,w\rangle$ is a $2$-dimensional subspace sitting inside $V$, so it splits off. Thus we get a hyperbolic form $\begin{pmatrix}0 & 1 \\ 1 & 0\end{pmatrix}$ (or something like this). (Example: torus)
\end{enumerate}
Thus we conclude:
\begin{prop}
Any nonsingular symmetric bilinear form on a finite dimensional vector space over $\FF_2$ is isomorphic to: $\begin{pmatrix}\begin{pmatrix}1 & \cdots & 0\\ \vdots & \ddots & \vdots \\ 0 & \cdots & 1 \end{pmatrix} & 0 \\ 0 & \begin{pmatrix}\begin{pmatrix}0 & 1 \\ 1 & 0\end{pmatrix} & \\ & \begin{pmatrix}0 & 1 \\ 1 & 0\end{pmatrix}\end{pmatrix}\end{pmatrix}$ where the bottom left thing has some number of the hyperbolic $\begin{pmatrix}0 & 1 \\ 1 & 0\end{pmatrix}$.
\end{prop}
Let $\mathbf{Bil}$ be the set of nonsingular symmetric bilinear forms that are finite dimensional over $\FF_2$ modulo isomorphisms. I've just given a classification of these things. This is a commutative monoid under $\bigoplus$. This $\mathbf{Bil}$ is $\{\text{nonsingular matrices}\}/\text{similarity}$ where similarity means $M\sim N$ if $N=AMA^T$ for some nonsingular $A$.
\begin{claim}
\begin{equation*}
\begin{pmatrix}
 & 1 & \\
1 & & \\
 & & 1
\end{pmatrix}
\sim
\begin{pmatrix}
1 & & \\
& 1 & \\
& & 1
\end{pmatrix}
\end{equation*}
This is the same thing as saying that $\begin{pmatrix}
 & 1 & \\
1 & & \\
 & & 1
\end{pmatrix}=AA^T$ for some nonsingular $A$.
\end{claim}
\begin{proof}
Let $A=\begin{pmatrix}1 & 1 & 1 \\ 1 & 0 & 1 \\ 0 & 1 & 1 \end{pmatrix}$.
\end{proof}
So, $\mathbf{Bil}$ is the commutative monoid generated by $(1)$, $\begin{pmatrix}0 & 1 \\ 1 & 0\end{pmatrix}$ with relation $\begin{pmatrix}
 & 1 & \\
1 & & \\
 & & 1
\end{pmatrix}
\sim
\begin{pmatrix}
1 & & \\
& 1 & \\
& & 1
\end{pmatrix}$.
\subsection{Connected surfaces}
Let's go back to topology. Let $n=2$, $k=1$ (so that $2k=n$). Then you get an intersection pairing on $ H_1(M)$. Suppose I consider $\RP^2$. We know that $ H_1(\RP^2)=\FF_2$. This must be that ``$(1)$''-form. This says that anytime you have a cycle on a projective plane, there's nothing I can do to remove its self interesections. This is also seen on a Mobius band that I will not draw because I don't know how to draw fundamental polygons in LaTeX. But this can be seen by looking at the projection from the Mobius band to the circle.

We also had the notion of connected sums. Let's compute the homology of $\Sigma_1\#\Sigma_2$ using Mayer-Vietoris, like you did in the exercise. You get $ H_2(\Sigma_1\#\Sigma_2)\to H_1(S^1)\to H_1(\Sigma_1-D^2)\oplus H_1(\Sigma_2-D^2)\to H_1(\Sigma_1\#\Sigma_2)$. When you pierce a surface, it collapses down into its $1$-skeleton, so $ H_2(\Sigma_1-D^2)\oplus H_2(\Sigma_2-D^2)\cong 0$. Let's restrict ourselves to connected surfaces. The map $ H_1(S^1)\to H_1(\Sigma_1-D^2)\oplus H_1(\Sigma_2-D^2)$ is the zero map when you think of it as a boundary. Actually, that argument needs to use something, because we know that with integer coefficients, this isn't right -- you need to find an oriented chain in that case -- but this orientation issue goes away in $\FF_2$ because there's no issues with $\pm$. And actually, the map $ H_1(\Sigma_1\#\Sigma_2)\to H_0(S^1)$ is zero as well, so $ H_1(\Sigma_1-D^2)\oplus H_1(\Sigma_2-D^2)\cong H_1(\Sigma_1)\oplus H_1(\Sigma_2)\cong H_1(\Sigma_1\#\Sigma_2)$ that respect the intersection pairing.

Let $\mathbf{Surf}$ be the commutative monoid of connected compact surfaces modulo homeomorphism. What we've proved is that there's a map $\mathbf{Surf}\to\mathbf{Bil}$ that takes connected sums to direct sums of orthogonal bilinear forms. And this map is an isomorphism. I.e.:
\begin{theorem}
There is an isomorphism of commutative monoids $\mathbf{Surf}\to\mathbf{Bil}$.
\end{theorem}
This is a little strange. What is our claim $\begin{pmatrix}
 & 1 & \\
1 & & \\
 & & 1
\end{pmatrix}
\sim
\begin{pmatrix}
1 & & \\
& 1 & \\
& & 1
\end{pmatrix}$ saying in this context? It's saying that $T^2\#\RP^2\cong(\RP^2)^{\# 3}$.
\begin{claim}
If $\Sigma$ is nonoriented, then $\Sigma\# T^2\cong\Sigma\# K$ (where $K$ is the Klein bottle $\RP^2\#\RP^2$).
\end{claim}
If it's nonoriented, then a Mobius strip $M$ sits into $\Sigma$. Uh I guess you're dragging your attachment of the torus along $M$?

There's more to be said about this. You can think of quadratic forms if you're working over the reals or something. You can't do this modulo $2$. Under the story we talked about, the oriented surfaces are the ones where the dot product is always zero. A better thing to ask over $\FF_2$ are quadratic forms such that $q(x+y)=q(x)+q(y)+x\cdot y$. This leads to the Kervaire invariant, etc.

Happy Thanksgiving! See you on Monday.
