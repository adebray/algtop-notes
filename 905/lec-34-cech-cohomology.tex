\section{$\cHH^\ast$ as a cohomology theory}
Office hours: today, Hood in 4-390 from 1:30 to 3:30 and Miller in 4-478 from 1-3 on Tuesday. Note that pset 6 is due Wednesday. Also, Wednesday we'll have a lightning review of $\pi_1$ and covering spaces.

We're coming to the end of the course, and there are going to be oral exams. I have some questions that I'd like to ask you. They won't be super advanced, detailed questions -- they'll be basic things. I'll post a list of examples of questions. I won't select questions from that list, that's cruel and isn't the point. The oral will be 40 minutes. It'll be fun -- better than a written exam. It's much better than grading a written exam!

PLEASE DRAW A PICTURE WHEN READING THIS IF YOU DIDN'T COME TO CLASS!
\subsection{$\cHH^\ast$ versus $H^\ast$}
Let's come up with an example that distinguishes $\cHH^\ast$ and $H^\ast$. This is a famous example -- the topologist's sine curve. The topologist's sine curve is defined as follows. Consider the graph of $\sin(\pi/x)$ where $0<x\leq 1$, where we draw a continuous curve from $(0,-1)$ to $(1,0)$. This is a counterexample for a lot of things, you've probably seen it in 18.901.

What is $H_\ast$ of the topologist sine curve? Use Mayer-Vietoris! I can choose the bottom half to be some connected portion of the continuous curve from $(0,-1)$ to $(1,0)$, and the top half to be the rest of the space that intersects the bottom half (in two spots). The top is the union of two path components, each contractible.

To see this, suppose I have $\sigma:I\to\mathrm{top}$ such that $\sigma(0)=(0,b)$ for some $-3/2 < b\leq 1$ (the $-3/2$ is arbitrary, just choose something $\leq 0$). If $b<1$, pick $\epsilon>0$ such that if we write $\sigma=(\sigma_1,\sigma_2)$, then $\sigma_2(t)<1$ for all $t\in[0,\epsilon)$, where we use continuity. Then $\sigma|_{[0,\epsilon)}$ can't be on the sine curve. If $b=1$, pick $\epsilon>0$ such that $\sigma_2(t)>-1$ for all $t\in[0,\epsilon)$. Similarly, it can't be on the sine curve.

Thus, $H_1$ is simple: it fits into $0\to H_1(X)\to H_0(\mathrm{top}\cap\mathrm{bottom})\to H_0(\mathrm{top})\oplus H_0(\mathrm{top})\to H_0(X)\to 0$. This is basically $0\to H_1(X)\xrightarrow{\partial}\to\Z\oplus\Z\hookrightarrow\Z\oplus\Z\oplus\Z\to\Z\to 0$, so $\partial=0$. This means that $H_\ast(X)\cong H_\ast(\ast)$ and this implies that $H^\ast(X)\cong H^\ast(\ast)$.

How about $\cHH^\ast$? Let $X\subset U$ be an open neighborhood. The interval is contained in some $\epsilon$-neighborhood that's contained in $U$. This implies that there exists a neighborhood $X\subseteq V\subseteq U$ such that $V\sim S^1$. Therefore, $\varinjlim_{U\in\mathcal{U}_X}H^\ast(U)\cong H^\ast(S^1)$ by ``cofinality''. So $\cHH^\ast$ and $H^\ast$ differ.
\subsection{Cofinality}
Let $\cI$ be a directed set. Let $A:\cI\to \mathbf{Ab}$ be a functor. If I have a functor $f:\cK\to\cI$, then I get $Af:\cK\to\mathbf{Ab}$, i.e., $(Af)_j=A_{f(j)}$.

I can form $\varinjlim_{\cK}Af$ and $\varinjlim_{I}A$. I claim you have a map $\varinjlim_{\cK}Af\to\varinjlim_{\cI}A$. All I have to do is the following:
\begin{equation*}
\xymatrix{
	\varinjlim_{J}Af\ar[r] & \varinjlim_{I}A\\
	A_{f(j)}\ar[u]^{\mathrm{in}_j}
}
\end{equation*}
So I have to give you maps $A_{f(j)}\to\varinjlim_{I}A$ for various $j$. I know what to do, because I have $\mathrm{in}_{f(j)}:A_{f(j)}\to\varinjlim_{I}A$. Are they compatible when I change $j$? Suppose I have $j^\prime\leq j$. Then I get a map $f(j^\prime)\to f(j)$, so I have a map $A_{f(j^\prime)}\to A_{f(j)}$, and thus the maps are compatible. Hence I get:
\begin{equation*}
\xymatrix{
	\varinjlim_{J}Af\ar[r] & \varinjlim_{I}A\\
	(Af)_j=A_{f(j)}\ar@{-->}[ur]^{\mathrm{in}_{f(j)}}\ar[u]^{\mathrm{in}_j}
}
\end{equation*}
\begin{example}
Suppose $K\supseteq L$ be closed, then I get a map $\cHH^\ast(K)\to\cHH^\ast(L)$. Is this a homomorphism? Well, $\cHH^\ast(K)=\varinjlim_{U\in\mathcal{U}_K}H^\ast(U)$ and $\cHH^\ast(L)=\varinjlim_{V\in\mathcal{U}_L}H^\ast(V)$. This is an example of a $\cI$ and $\cK$ that I care about. Well, $\mathcal{U}_K\subseteq\mathcal{U}_L$, and thus I get a map $\cHH^\ast(K)\to\cHH^\ast(L)$, which is what I wanted.

I can do something for relative cohomology. Suppose:
\begin{equation*}
\xymatrix{K\ar@{^(->}[d] & L\ar@{^(->}[d]\ar@{_(->}[l] \\ K^\prime & L^\prime\ar@{_(->}[l]}
\end{equation*}
I get a homomorphism $\cHH^\ast(K,L)\to \cHH^\ast(K^\prime,L^\prime)$ because I have $\mathcal{U}_{K,L}\to\mathcal{U}_{K^\prime,L^\prime}$.
\end{example}
This isn't exactly what we need:
\begin{question}
When does $f:\cK\to\cI$ induce an isomorphism $\varinjlim_{J}Af\to\varinjlim_{I}A$?
\end{question}
This is a lot like taking a sequence and a subsequence and asking when they have the same limit. There's a cofinality condition in analysis, that has a similar expression here.
\begin{definition}
$f:\cK\to\cI$ is cofinal if for all $i\in\cI$, there exists $j\in\cK$ such that $i\leq f(j)$.
\end{definition}
\begin{example}
If $f$ is surjective.
\end{example}
\begin{lemma}
If $f$ is cofinal, then $\varinjlim_{J}Af\to\varinjlim_{I}A$ is an isomorphism.
\end{lemma}
\begin{proof}
Check that $\{A_{f(j)}\to\varinjlim_{I}A\}$ satisfies the necessary and sufficient conditions:
\begin{enumerate}
\item For all $a\in\varinjlim_{I}A$, there exists $j$ and $a_j\in A_{f(j)}$ such that $a_j\mapsto a$. We know that there exists some $i$ and $a_i\in A$ such that $a_i\mapsto a$. Pick $j$ such that $f(j)\geq i$, so we get a map $a_i\to a_{f(j)}$, and by compatibility, we get $a_{f(j)}\mapsto a$.
\item The other condition is also just as easy.
\end{enumerate}
\end{proof}
This is a very convenient condition.
\begin{example}
I had a perverse way of constructing $\QQ$ by using the divisibility directed system. A much simpler (linear!) directed system is $\Z\xrightarrow{2}\Z\xrightarrow{3}\Z\xrightarrow{4}\Z\to\cdots$. This has the same colimit as the divisibility directed system because $n|n!$, so we have a cofinal map between directed systems.
\end{example}
How about the direct limits in the \v{C}ech cohomology case?
\begin{example}
Do I have a map $\cHH^\ast(K,L)\to\cHH^\ast(K)$? Suppose:
\begin{equation*}
\xymatrix{K\ar@{^(->}[d] & L\ar@{^(->}[d]\ar@{_(->}[l] \\ U & V\ar@{_(->}[l]}
\end{equation*}
Then $\cHH^p(K,L)=\varinjlim_{(U,V)\in\mathcal{U}_{K,L}}H^p(U,V)$ and $\cHH^p(K)=\varinjlim_{U\in\mathcal{U}_K}H^p(U)$. I have a map of directed sets $\mathcal{U}_{K,L}\to\mathcal{U}_K$ by sending $(U,V)\mapsto U$. I didn't have to use cofinality. I want a long exact sequence, though, and I'm going to do this by saying that it's a directed limit of a long exact sequence. I'm going to have to have all of these various \v{C}ech cohomologies as being the directed limit over the \emph{same} indexing set.

I'd really like to say that $\cHH^p(K)=\varinjlim_{U\in\mathcal{U}_K}H^p(U)\cong \varinjlim_{(U,V)\in\mathcal{U}_{K,L}}H^p(U)$. Thus I need to show that $\mathcal{U}_{K,L}\to\mathcal{U}_K$ where $(U,V)\mapsto U$ is cofinal. This is easy, because if $U\in\mathcal{U}_K$, just pick $(U,U)$, i.e., $\mathcal{U}_{K,L}\to\mathcal{U}_K$ is cofinal. How about $\mathcal{U}_{K,L}\to\mathcal{U}_L$ by $(U,V)\mapsto V$; is it cofinal? Yes! For $V\in\mathcal{U}_L$, pick $(X,V)$! This means that $\cdots\cHH^{p-1}(L)\to\cHH^p(K,L)\to\cHH^p(K)\to\cHH^p(L)\to\cHH^{p+1}(K,L)$ is $\varinjlim_{\mathcal{U}_{K,L}}\left(\cdots\to H^{p}(U,V)\to\cdots\right)$, and hence exact.
\end{example}
How about excision? I need this to get to Mayer-Veitoris!
\begin{lemma}
Assume $X$ is normal and $A,B$ are closed subsets. Then $\cHH^p(A\cup B,B)\to\cHH^p(A,A\cap B)$ is an isomorphism. 
\end{lemma}
\begin{proof}
Well, $\cHH^p(A\cup B,B)$ is $\varinjlim$ over $\mathcal{U}_{A\cup B,B}$ and $\cHH^p(A,A\cap B)$ is $\varinjlim$ over $\mathcal{U}_{A,A\cap B}$. Let $W\supseteq A$ and $Y\supseteq B$ are neighborhoods. I claim that $\mathcal{U}_A\times\mathcal{U}_B\to\mathcal{U}_{A\cup B,B}$ sending $(W,Y)\mapsto (W\cup Y,Y)$ and $\mathcal{U}_A\times\mathcal{U}_B\to\mathcal{U}_{A,A\cap B}$ sending $(W,Y)\mapsto (W,W\cap Y)$ are cofinal.

If I give you $(U,V)\in \mathcal{U}_{A\cup B,B}$, define $(W,V)\in\mathcal{U}_A\times\mathcal{U}_B$ where $W=U$ and $Y=V$, so $\mathcal{U}_A\times\mathcal{U}_B\to\mathcal{U}_{A\cup B,B}$ is surjective, hence cofinal. The latter is trickier. Let $U\supseteq A$ and $V\supseteq A\cap B$. Here's where normality comes into play. Separate $B-V$ from $A$. Let $T\supseteq B-V$. Shit. \emph{Shit!}

Maybe I'll leave this to you. I'll put this on the board on Wednesday. Anyway, I'll use normality to show that $\mathcal{U}_A\times\mathcal{U}_B\to\mathcal{U}_{A,A\cap B}$ is cofinal, and thus this verifies excision -- so you actually have excision.
\end{proof}
