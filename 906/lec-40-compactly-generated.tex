\section{Compactly generated spaces}
A lot of homotopy theory is about loop spaces and mapping spaces.
Standard topology doesn't do very well with mapping spaces, so we will narrate the story of \emph{compactly generated spaces}.
One nice consequence of working with compactly generated spaces is that the category is
Cartesian-closed (a concept to be defined below).

\subsection{CGHW spaces}
Some constructions commute for ``categorical reasons''.
For instance, limits commute with limits.
Here is an exercise to convince you of a special case of this.
\begin{exercise}%Exercise 4 from 18.906
    Let $X$ be an object of a category $\cc$.
    The \emph{overcategory} (or the \emph{slice category}) $\cc_{/X}$ has objects
    given by morphisms $p:Y\to X$ in $\cc$, and morphisms given by the obvious
    commutativity condition.
    \begin{enumerate}
	\item Assume that $\cc$ has finite products.
	    What is the left adjoint to the functor $X\times -:\cc\to\cc_{/X}$ that sends
	    $Y$ to the object $X\times Y \xar{\pr_1}X$?
	\item As a consequence of Theorem \ref{adjointslimits}, we find that $X\times -:\cc\to\cc_{/X}$ preserves limits.
	    The composite $\cc\to\cc_{/X}\to\cc$, however, probably does not.
	    \begin{itemize}
		\item What is the limit of a diagram in $\cc_{/X}$?
		\item Let $Y:\cI\to\cc$ be any diagram. Show that
		    $${\lim_{i\in\cI}}^{\cc_{/X}}(X\times Y_i)
		    \simeq X\times {\lim_{i\in\cI}}^\cc Y_i.$$
		    What happens if $\cI$ only has two objects and only identity morphisms?
	    \end{itemize}
    \end{enumerate}
\end{exercise}
However, colimits and limits need not commute!
An example of colimits failing to commute with limits comes from algebra.
The coproduct in the category of commutative rings is the tensor product.
But $\left(\lim \Z/p^k\Z\right) \otimes \QQ \simeq \Z_p \otimes \QQ \simeq \QQ_p$ is
clearly not $\lim\left(\Z/p^k\Z\otimes \QQ\right) \simeq \lim 0 \simeq 0$!

We also need not have an isomorphism between
$X\times\colim_{j\in \cJ}Y_j$ and $\colim_{j\in \cJ}(X\times Y_j)$:
one such example comes a quotient map $Y\to Z$: 
in general, the induced map $X\times Y\to X\times Z$ is not necessarily another quotient map.
A theorem of Whitehead's says that this problem is rectified if we assume that $X$ is
a compact Hausdorff space.
Unfortunately, a lot of interesting maps are built up from more ``elementary'' maps by such a procedure,
so we would like to repair this problem.
%Why were we talking about colimits? Here's an observation. Suppose $X\to Y$ is a quotient map; then a map $Y\to Z$ is continuous iff the composite $X\to Y\to Z$ is continuous. A quotient map \emph{is} a coequalizer. What I'm saying is, I can find two maps to $X$ such that $Y$ is a coequalizer of $X$. What space are we mapping into $X$? Well, suppose $Z=X/\sim$. If we considered:
%\begin{equation*}
%    \begin{tikzcd}
%	X\times_Z X\ar[r,shift left=.75ex,"\pi_1"]\ar[r,shift right=.75ex,swap,"\pi_2"] & X\ar[r] & Z
%    \end{tikzcd}
%\end{equation*}
%The term here is ``regular epimorphism''.

We cannot simply do this by restricting ourselves to compact Hausdorff spaces:
that's a pretty restrictive condition to place.
Instead (motivated partially by the Yoneda lemma),
we will look at topologies detected by maps from compact Hausdorff spaces.
\begin{definition}
    Let $X$ be a space.
    A subspace $F\subseteq X$ is said to be \emph{compactly closed} if,
    for any map $k:K\to X$ from a compact Hausdorff space $K$, 
    the preimage $k^{-1}(F)\subseteq K$ is closed.
\end{definition}
It is clear that any closed subset is compactly closed, but
there might be compactly closed sets which are not closed in the topology on $X$.
This motivates the definition of a $k$-space:
\begin{definition}
    A topological space $X$ is said to be a $k$-space
    if every compactly closed set is closed.
\end{definition}
The $k$ comes either from ``kompact'' and/or Kelly, who was an early topologist
who worked on such foundational topics.

Let $X$ be any space. It can be $k$-ified to some space denoted $kX$. You just enlarge the topology so that it includes the compactly closed sets. This is a topology, bigger than the original one. So the identity $kX\to X$ is continuous.

\begin{remark}
    $X$ is a $k$-space iff it has the property that a map $X\to Y$ is continuous iff for any compact Hausdorff $K$ and map $k:K\to X$, the composite $K\to X\to Y$ is continuous.
\end{remark}
\begin{example}
    Compact Hausdorff spaces are $k$-spaces. First countable (so metric spaces) and CW-complexes are also $k$-spaces.
\end{example}
Define $k\Top$ to be the category of $k$-spaces. There's an inclusion $i:k\Top\hookrightarrow \Top$. Now $k$-ification gives a functor $\Top\to k\Top$. This has the property that:
$$k\Top(X,kY)=\Top(iX,Y)$$
Here's an adjunction! This means that $k(iX\times iY)=X\times^{k\Top}Y$ where $X$ and $Y$ are $k$-spaces. It's true that $kiX\simeq X$.

The takeaway is that $k\Top$ has good categorical properties inherited from $\Top$, i.e. it's complete and cocomplete. I want to now talk about mapping spaces, which shows that $k\Top$ is even better!
\subsection{Mapping spaces}
Let $X$ and $Y$ be spaces. There's the compact-open topology on $\Top(X,Y)$. For $k$-spaces, I want to make a slight modification. In particular, if $X$ and $Y$ are $k$-spaces, define a topology on $k\Top(X,Y)$ generated by: $W(k:K\to X, \text{ open }U\subseteq Y)=\{f:X\to Y: f(k(K))\subseteq U\}$. We write $Y^X$ for the $k$-ification of this space.
\begin{prop}
    \begin{enumerate}
	\item $(k\Top)^{op}\times k\Top\to k\Top$ given by $(X,Y)\to Y^X$ is a functor of both variables.
	\item $e:X\times Z^X\to Z$ given by $(x,f)\mapsto f(x)$ and $i:Y\to (X\times Y)^X$ given by $y\mapsto(x\mapsto(x,y))$ are continuous.
    \end{enumerate}
\end{prop}
\begin{proof}
    Look online -- there are references on the webpage.
\end{proof}
I'm not where I wanted to be at this moment in time. Ok, here's a result of this proposition. Consider $k\Top(X\times Y,Z)$ where the product is the product in $k$-spaces. What is $k\Top(Y,Z^X)$? I want to say that:
$$k\Top(X\times Y,Z)\simeq k\Top(Y,Z^X)$$
defined by $(f:X\times Y\to Z)\mapsto (Y\xrightarrow{i}(X\times Y)^X\to Z^X)$ in one direction, and by $(f:Y\to Z^X)\mapsto(X\times Y\to X\times Z^X\xrightarrow{e} X)$. They're so natural that they have to be inverses to one another. Let me close with a definition.
\begin{definition}
    A category $\cc$ with finite products is Cartesian closed if for any $X$, the functor $X\times -:\cc\to \cc$ has a right adjoint.
\end{definition}
So $k\Top$ is Cartesian closed, but $\Top$ isn't. On Monday, I'll justify why this is important.
