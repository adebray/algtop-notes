\section{Fiber bundles, fibrations, cofibrations}
Having set up the requisite technical background, 
we can finally launch ourselves from point-set topology to the world of homotopy theory.
\subsection{Fiber bundles}
\begin{definition}\label{fiberbundle}
    A fiber\footnote{Or ``fibre'', if you're British.} bundle is a map $p:E\to B$,
    such that for every $b\in B$, there exists:
    \begin{itemize}
	\item an open subset $U\subseteq B$ that contains $b$, and
	\item a map $p^{-1}(U)\to p^{-1}(b)$ such that $p^{-1}(U)\to U\times p^{-1}(b)$ is a homeomorphism.
    \end{itemize}
    If $p:E\to B$ is a fiber bundle, $E$ is called the \emph{total space}, $B$ is called the
    \emph{base space}, $p$ is called a \emph{projection},
    and $F$ (sometimes denoted $p^{-1}(b)$) is called the \emph{fiber over $b$}.
\end{definition}
In simpler terms: the preimage over every point in $B$ looks like a product,
i.e., the map $p:E\to B$ is ``locally trivial'' in the base.

Here is an equivalent way of stating Definition \ref{fiberbundle}:
there is an open cover $\cU$ (called the \emph{trivializing cover}) of $B$,
such that for every $U\subseteq \cU$,
there is a space $F$, and a homeomorphism
$p^{-1}(U)\simeq U\times F$ that is compatible with the projections down to $U$.
(So, for instance, a trivial example of a fiber bundle is just the projection map $B\times F\xar{\mathrm{pr}_1} B$.)

Fiber bundles are naturally occuring objects.
For instance, a covering space $E\to B$ is a fiber bundle with discrete fibers. 
\begin{example}[The Hopf fibration]
    The Hopf fibration is an extremely important example of a fiber bundle.
    Let $S^3\subset \cC^2$ be the $3$-sphere.
    There is a map $S^3\to \CP^1 \simeq S^2$ that is given by sending a vector $v$ to the
    complex line through $v$ and the origin.
    This is a non-nullhomotopic map, and is a fiber bundle whose fiber is $S^1$.
\end{example}
The Hopf fibration is a map between smooth manifolds.
A theorem of Ehresmann's says that it is not too hard to construct fiber bundles over smooth manifolds:
\begin{theorem}[Ehresmann]
    Suppose $E$ and $B$ are smooth manifolds, and let $p:E\to B$ be a smooth (i.e., $C^\infty$) map.
    Then $p$ is a fiber bundle if:
    \begin{enumerate}
	\item $p$ is a \emph{submersion}, i.e., $dp:T_e E\to T_{p(e)} B$ is a surjection, and
	\item $p$ is \emph{proper}, i.e., preimages of compact sets are compact.
    \end{enumerate}
\end{theorem}
%For example, I can look at the complement of a closed set in $S^3$, and then the restriction of $p$ won't be a fiber again.
The purpose of this part of the book is to understand fiber bundles through algebraic methods like cohomology and homotopy.
This means that we will usually need a ``niceness'' condition on the fiber bundles that we will be studying;
this condition is made precise in the following definition (see \cite{MayConcise}).
\begin{definition}
    Let $X$ be a space.
    An open cover $\cU$ of $X$ is said to be \emph{numerable}
    if there exists a subordinate partition of unity, i.e.,
    for each $U\in\cU$, there is a function $f_U:X\to [0,1]=I$
    such that $f^{-1}((0,1]) = U$,
    and any $x\in X$ belonds to only finitely many $U\in\cU$.
    The space $X$ is said to be \emph{paracompact} if any open cover admits a numerable refinement.
\end{definition}
This isn't too restrictive for us algebraic topologists since CW-complexes are paracompact.
\begin{definition}
    A fiber bundle is said to be \emph{numerable} if it admits a numerable trivializing cover.
\end{definition}

\subsection{Fibrations and path liftings}
For our purposes, though, fiber bundles are still too narrow.
Fibrations capture the essence of fiber bundles, although it is not at all immediate from their definition that this is the case!
\begin{definition}\label{fibration}
    A map $p:E\to B$ is called a \emph{(Hurewicz\footnote{Named after Witold Hurewicz, who was one of the first algebraic
    topologists at MIT.}) fibration} if it satisfies the
    \emph{homotopy lifting property} (commonly abbreviated as HLP):
    suppose $h:I\times W\to B$ is a homotopy; then there exists a lift\footnote{Note that we place no restriction on the
    \emph{uniqueness} of this lift.}
    (given by the dotted arrow) that makes the diagram commute:
    \begin{equation}\label{hlp}
	\xymatrix{
	    W\ar[r]^f\ar[d]_{\mathrm{in}_0} & E\ar[d]^p\\
	    I\times W\ar[r]_h\ar@{-->}[ur]^{\overline{h}} & B,
	    }
    \end{equation}
\end{definition}
At first sight, this seems like an extremely alarming definition, since
the HLP has to be checked for \emph{all} spaces, \emph{all} maps, and \emph{all} homotopies! 
The HLP is not impossible to check, though.

\begin{exercise}\label{productfibration}
    Check that the projection $\mathrm{pr}_1: B\times F\to B$ is a fibration.
\end{exercise}

\begin{exercise}
    Check the following statements.
    \begin{itemize}
	\item Fibrations are closed under pullbacks. In other words, if $p:E\to B$ is a fibration and $X\to B$ is any map, then the induced map $E\times_B X\to X$ is a fibration.
	\item Fibrations are closed under exponentiation and products. In other words, if $p:E\to B$ is a fibration, then $E^A\to B^A$ is another fibration.
	\item Fibrations are closed under composition.
    \end{itemize}
\end{exercise}

\begin{exercise}
    Let $p:E_0 \to B_0$ be a fibration, and let $f:B \to B_0$ be a homotopy equivalence.
    Prove that the induced map $B\times_{B_0} E_0 \to E_0$ is a homotopy equivalence.
    (Warning: this exercise has a lot of technical details! The end of this chapter describes an
    alternative\footnote{``Alternative'' in the sense that the proof uses statements
    not covered yet in this book.}
    solution to this exercise, when $E_0$ and $B\times_{B_0} E_0$ are CW-complexes.)
    \todo{Don't forget to do this!}
\end{exercise}


There is a simple geometric interpretation of what it means for a map to be a fibration, in terms of ``path liftings''.
To understand this description, we will reformulate the diagram \eqref{hlp}.
Given that we are working in the category of CGWH spaces, one of the first things we can attempt to do is adjoint the $I$;
this gives the following diagram.
\begin{equation}\label{hlp2}
    \xymatrix{
	E\ar[r]^p & B\\
	W\ar[u]^f\ar[r]_{\widehat{h}} & B^I\ar[u]_{\mathrm{ev}_0}
    }
\end{equation}
By the definition of the pullback of a diagram, the data of this diagram is equivalent to a map $W\to B^I\times_B E$.
Explicitly,
$$B^I\times_B E = \{(\omega, e) \in B^I\times E \text{ such that } \omega(0) = p(e)\}.$$

Suppose the desired dotted map exists (i.e., $p:E\to B$ satisfied the HLP).
This would beget (again, by adjointness) a lifted homotopy $\widehat{\overline{h}}:W\to E^I$.
Since we already have a map\footnote{Clearly $(p\omega)(0) = p(\omega(0))$, so this map is well-defined
(i.e., the image lands in $B^I\times_B E$).} $\widetilde{p}:E^I\to B^I\times_B E$ given by $\omega\mapsto (p\omega,\omega(0))$,
the existence of the lift $\overline{h}$ in the diagram \eqref{hlp} is equivalent to the existence of a lift in
the following diagram.
\begin{equation*}
    \xymatrix{
	& E^I\ar[d]^{\widetilde{p}}\\
	W\ar[r]\ar@{-->}[ur]^{\widehat{\overline{h}}} & B^I\times_B E
    }
\end{equation*}
Obviously the universal example of a space $W$ that makes the diagram \eqref{hlp2} commute is $B^I\times_B E$ itself.
If $p$ is a fibration, we can make the lift in the following diagram.
%and if I can lift for any $W$, I can obviously construct the lift in the following diagram:
%idk why the above line was typed in
\begin{equation*}
    \xymatrix{
	& E^I\ar[d]^{\widetilde{p}}\\
	B^I\times_B E\ar@{-->}[ur]^\lambda\ar[r]^1 & B^I\times_B E
    }
\end{equation*}
The map $\lambda$ is called a \emph{lifting function}.
To understand why, suppose $(\omega, e) \in B^I\times_B E$, so that $\omega(0) = p(e)$.
In this case, $\lambda(\omega,e)$ defines a path in $E$ such that
$$p\circ\lambda(\omega, e) = \omega,\text{ and }\lambda(\omega,e)(0) = e.$$
Taking a step back and assessing the situation, we find that the lifting function $\lambda$ starts with a path
$\omega$ in $B$, and some point in $E$ mapping down to $\omega(0)$,
and produces a ``lifted'' path in $E$ which lives over $\omega$.
In other words, the map $\lambda$ is a path lifting: it's a continuous way to lift paths in the base space $B$ to
the total space $E$. 

The following result is a ``consistency check''.
\begin{theorem}[Dold]
    Let $p:E\to B$ be a map. Assume there's a numerable cover of $B$, say $\cU$, such that for every $U\in\cU$,
    the restriction $p|_{p^{-1}(U)}:p^{-1}U\to U$ is a fibration.
    (In other words, $p$ is \emph{locally} a fibration over the base).
    Then $p$ itself is a fibration.
\end{theorem}
In particular, one consequence of this theorem is that every numerable fiber bundle is a fibration.
Our discussion above tells us that numerable fiber bundles satisfy the homotopy (and hence path) lifting property.
This is great news, as we will see shortly.
