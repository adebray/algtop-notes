\section{Homotopy fibers}
An important, but easy, fact about fibrations is that the canonical map $X\to \ast$ from any space $X$ is a fibration
\footnote{Model category theorists get excited about this, because this says that all objects in the associated model structure on
topological spaces is fibrant.}.
This is because the dotted lift in the diagram below can be taken to the map $(t,w)\mapsto f(w)$:
\begin{equation*}
    \xymatrix{
    W\ar[d]\ar[r]^f & X\ar[d]\\
    I\times W\ar@{-->}[ur]\ar[r] & \ast.
}
\end{equation*}
However:
\begin{exercise}
    The inclusion $\ast\hookrightarrow X$ is not always a cofibration;
    if it is, say that $\ast$ is a \emph{nondegenerate basepoint} of $X$.
    Give an example of a compactly generated space $X$ for which this is true.
\end{exercise}
If $\ast$ has a neighborhood in $X$ that contracts to $\ast$,
the inclusion $\ast\hookrightarrow X$ is a cofibration.
Note that if $\ast$ is a nondegenerate basepoint, the canonical map $X^A\xrightarrow{\mathrm{ev}} X$ is a fibration,
where $A$ is a pointed subspace of $X$ (with basepoint given by $\ast$).
The fiber of $\mathrm{ev}$ is exactly the space of pointed maps $A\to X$.

\begin{remark}
    In Example \ref{intervalcofib}, we saw that $\{0,1\}\hookrightarrow I$ is a cofibration;
    this implies that the map $Y^I\to Y\times Y$ (given by $\omega\mapsto (\omega(0),\omega(1))$) is a fibration.
\end{remark}
\subsection{``Fibrant replacements''}
The purpose of this subsection is to provide a proof of the following result, which says that every map can be ``replaced''
(up to homotopy) by a fibration.
\begin{theorem}\label{fibrep}
    For any map $f:X\to Y$, there is a space $T(f)$, along with a fibration $p:T(f) \to Y$ and
    a homotopy equivalence $X \xar{\simeq} T(f)$, such that the following diagram
    commutes:
    \begin{equation*}
	\xymatrix{
	    X\ar[r]^{\simeq}\ar[dr]_f & T(f)\ar[d]^p\\
	    & Y.
	    }
    \end{equation*}
\end{theorem}
\begin{proof}
    Consider the map $Y^I\xrightarrow{\binom{\ev_0}{\ev_1}}Y\times Y$.
    Let $T(f)$ be the pullback of the following diagram:
	\begin{equation*}
	    \xymatrix{
	T(f)\ar[r]\ar[d] & Y^I\ar[d]^{\binom{\ev_0}{\ev_1}}\\
		    X\times Y\ar[r]_{f\times 1} & Y\times Y.
		}
	\end{equation*}
	So, as a set, we can write
	$$T(f)=\{(x,\omega)\in X\times Y^I|f(x) = \omega(0)\}.$$

	Let us check that the canonical map $T(f) \to Y$, given by $(x,\omega) \mapsto \omega(1)$, is a fibration.
	The projection map $\pr: X\times Y\to Y$ is a fibration, so it suffices to show that the map $T(f) \to X\times Y$
	is also a fibration.
	Since fibrations are closed under pullbacks, we are reduced to checking that the map $Y^I\to Y\times Y$ is a fibration;
	but this is exactly saying that the inclusion $\{0,1\} \hookrightarrow I$ is a cofibration, which it is
	(Example \ref{intervalcofib}).

	To prove that $X$ is homotopy equivalent to $T(f)$, we need to produce a map $X\to T(f)$.
	This is equivalent to giving maps $X\to X\times Y$ and $X\to Y^I$ that have compatible images in $Y\times Y$.
	The first map can be chosen to be $X\xrightarrow{\binom{1}{f}}X\times Y$.
	Define the map $X\to Y^I$ by sending $x\in X$ to the constant loop at $f(x)$.
	It is clear that both composites $X\to X\times Y\to Y\times Y$ and $X\to Y^I\to Y\times Y$ are the same;
	this defines a map $X\to T(f)$, denoted $g$.
	As one can easily check, the composite $X\to T(f)\xar{p} Y$ is the map $f:X\to Y$ that we started off with.
	It remains to check that this map $X\xar{g} T(f)$ is a homotopy equivalence.
	We will construct a homotopy inverse to this map.

	The composite $X\to T(f)\to X\times Y\to X$ is the identity, so one candidate for a homotopy inverse to $g$ is
	the composite
	$$T(f)\to X\times Y\xrightarrow{pr_1} X.$$
	To prove that this map is indeed a homotoyp inverse to $g$, we need to consider the composite
	$T(f)\to X\xar{g} T(f)$, which sends $(x,\omega)\mapsto x\mapsto (x,c_{f(x)})$ where,
	recall, $c_{f(x)}$ is the constant path at $x$.
	We need to produce a homotopy between this composite and the identity on $T(f)$.
	
	Let $s\in I$. Given $\omega \in Y^I$, define a new loop $\omega_s$ by $\omega_s(t) = \omega(st)$.
	For instance, $\omega_1 = \omega$, and $\omega_0 = c_{\omega(0)}$ --- so, the loop $\omega_s$ ``sucks in''
	the point $\omega(1)$.
	This is precisely what we need to produce a homotopy between the composite $T(f) \to X \xar{g} T(f)$ and
	$\mathrm{id}_{T(f)}$, since the only constraint on $(x,\omega)\in T(f)$ is on $\omega(0)$.
	The following map provides the desired homotopy equivalence $X \simeq T(f)$.
	\begin{align*}
	    H:I\times T(f)\to & T(f)\\
	    (s,(x,\omega))\mapsto & (x,\omega_s).
	\end{align*}
%If you want to work through the diagram, this is how it looks.
%\begin{equation*}
%    \xymatrix{
%        X\ar@{-->}[dr]\ar[drr]^{x\mapsto c_{f(x)}}\ar[ddr]_{\begin{pmatrix}1 \\ f\end{pmatrix}} & & \\
%    	& T(f)\ar[d]\ar[r]\ar[dr] & Y^I\ar[d]^{\begin{pmatrix}ev_0 \\ ev_1\end{pmatrix}}\\
%    	    X & X\times Y\ar[l]^{pr_1}\ar[d]^{pr_2}\ar[r] & Y\times Y\\
%        & Y
%        }
%\end{equation*}
\end{proof}

\begin{example}[Path-loop fibration]
    This is a silly, but important, example.
    If $X = \ast$, the space $T(f)$ consists of paths $\omega$ in $Y$ such that $\omega(0) = \ast$.
    In other words, $T(f) = Y^I_\ast$; this is called the \emph{(based) path space} of $Y$, and is denoted by $P(Y,\ast)$,
    or simply by $PY$.
    The fiber of the fibration $T(f) = PY \to Y$ consists of paths that begin and end at $\ast$, i.e., 
    loops on $Y$ based at $\ast$.
    This is denoted $\Omega Y$, and is called the \emph{(based) loop space} of $Y$.
    The resulting fibration $PY \to Y$ is called the \emph{path-loop fibration}.
\end{example}

\begin{exercise}[``Cofibrant replacements'']\label{cofibrep}
    In this exercise, you will prove the analogue of Theorem \ref{fibrep} for cofibrations.
    Let $f:X\to Y$ be any map.
    Show that $f$ factors (functorially) as a composite $X \to M \to Y$, where $X\to M$ is a cofibration and $M\to Y$ is a homotopy
    equivalence.
\end{exercise}

\begin{solution}
    Define $Mf$ via the pushout:
    \begin{equation*}
	\xymatrix{
	    X\ar[r]^f \ar[d]_{\mathrm{in}_0} & Y\ar[d]^g\\
	    I\times X\ar[r] & Mf.
	    }
    \end{equation*}
    Define $r:Mf\to Y$ via $r(y) = y$ on $Y$ and $r(x,s) = f(x)$ on $X\times I$.
    Then, clearly, $rg = \mathrm{id}_Y$.
    There is a homotopy $\mathrm{id}_{Mf}\simeq gr$ given by the map $h:Mf\times I\to Mf$,
    defined by the formulae
    $$h(y,t) = y,\text{ and }h((x,s),t) = (x,(1-t)s).$$
    %Essentially, one's pushing the cylinder $X\times I$ down.
    
    We now have to check that $X\to Mf$ is a cofibration, i.e., that
    $Mf\times I$ retracts onto $Mf\times\{0\}\cup_X (X\times I)$.
    This can be done by ``pushing'' $Y\times I$ to $Y\times\{0\}$ and $X\times I\times I$ down to $X\times I$,
    while fixing $X\times\{0\}$.
    
    It is easy to see that this factorzation is functorial:
    if $f:X\to Y$ is sent to $g:W\to Z$ via $p:X\to W$ and $q:Y\to Z$,
    then $Mf\to Mg$ can be defined as the dotted map in the following diagra
    (which exists, by the universal property of the pushout):
    \begin{equation*}
	\xymatrix{
	    X\ar[rr]^f\ar[dd]_{\mathrm{in}_0}\ar[dr]|p & & Y\ar[dd]|\hole\ar[dr]|q & \\
	    & W\ar[rr]_g\ar[dd] & & Z\ar[dd]\\
	    X\times I\ar[rr]\ar[dr]_{p\times \mathrm{id}} & & Mf\ar@{-->}[dr] & \\
	    & W\times I\ar[rr] & & Mg.
	    }
    \end{equation*}
    \todo{Fix the overlapping arrows here, I don't know how to do this...}
\end{solution}

\subsection{Homotopy fibers}
One way to define the fiber (over a basepoint) of a map $f:X \to Y$ is via the pullback
\begin{equation*}
    \xymatrix{
	f^{-1}(\ast)\ar[r]\ar[d] & X\ar[d]^f\\
	\ast\ar[r] & Y.
    }
\end{equation*}
If $g:W \to X$ is another map such that the composite $W\xar{g} X\xar{f} Y$ is trivial, the map $g$ factors through $f^{-1}(\ast)$.
In homotopy theory, maps are generally not trivial ``on the nose''; instead, we usually have a nullhomotopy of a map.
Nullhomotopies of composite maps do not factor through this ``strict'' fiber; this leads to the notion of a homotopy fiber.
\begin{definition}[Homotopy fiber]
    The \emph{homotopy fiber} of a map $f:X\to Y$ is the pullback:
    \begin{equation*}
	\xymatrix{
	    F(f,\ast)\ar[r]\ar[d] & T(f)\ar[r]^\simeq \ar[d]^p & X\ar[dl]^f\\
	    \ast \ar[r] & Y &
	    }
    \end{equation*}
\end{definition}
As a set, we have
\begin{equation}\label{sethomotopyfiber}
    F(f,\ast) = \{(x,\omega)\in X\times Y^I| f(x) = \omega(0), \omega(1) = \ast\}.
\end{equation}
A nullhomotopic composite $W\to X\xar{f} Y$ factors as $W\to F(f,\ast)\to X\xar{f} Y$.
\begin{warning}
    The ordinary fiber and the homotopy fiber of a map are generally not the same!
    There is a canonical map $p^{-1}(\ast) \to F(p,\ast)$, but it is generally not a homotopy equivalence.
\end{warning}
\begin{prop}\label{strictcomparison}
    Suppose $p:X\to Y$ is a fibration.
    Then the canonical map $p^{-1}(\ast)\to F(p,\ast)$ is a homotopy equivalence.
\end{prop}
You will prove this in a series of exercises.

\begin{exercise}
    Prove Proposition \ref{strictcomparison} by working through the following statements.
    \begin{enumerate}
	\item Let $p:E\to B$ be a fibration.
	    Suppose $g:X \to B$ lifts across $p$ up to homotopy, i.e., there exists a map $f:X\to E$ such that
	    $p \circ f \simeq g$.
	    Prove that there exists a map $f^\prime:X\to E$ that is homotopic to $f$, such that $p \circ f^\prime = g$
	    (on the nose).
%	\item Let $p:E_0 \to B_0$ be a fibration, and let $f:B \to B_0$ be a homotopy equivalence.
%	    Prove that the induced map $B\times_{B_0} E_0 \to E_0$ is a homotopy equivalence.
%	    (Warning: this exercise has a lot of technical details! The end of this chapter describes an
%	    alternative\footnote{``Alternative'' in the sense that the proof uses statements
%	    not covered yet in this book.}
%	    solution to this exercise, when $E_0$ and $B\times_{B_0} E_0$ are CW-complexes.)
%	    \todo{Don't forget to do this!}
	\item Show that if $p:E\to B$ and $p^\prime:E^\prime \to B$ are fibrations, and $f:E \to E^\prime$ such that
	    $p^\prime \circ f = p$, the map $f$ is a \emph{fiber homotopy equivalence}: there is a homotopy inverse
	    $g: E^\prime \to E$ such that $g$, and the two homotopies $fg \simeq \mathrm{id}_{E^\prime}$ and
	    $gf \simeq \mathrm{id}_{E}$ are all fiber preserving (e.g., $p\circ g = p^\prime$).
	\item Conclude Proposition \ref{strictcomparison}.
   \end{enumerate}
\end{exercise}

Before we proceed, recall that we constructed the homotopy fiber by replacing $f:X\to Y$ by a fibration.
In doing so, we implicitly made a choice: we could have replaced the map $\ast\to Y$ by a fibration.
Are the resulting pullbacks the same?

By replacing $\ast \to Y$ by a fibration (namely, the path-loop fibration), we end up with the following pullback diagram:
\begin{equation*}
    \xymatrix{
	F^\prime(f,\ast)\ar[r]\ar[d] & P(Y,\ast)\ar[r]^{\simeq}\ar[d] & \ast\ar[dl]\\
	X\ar[r]_f & Y & 
    }
\end{equation*}
As a set, we have
$$F^\prime(f,\ast) = \{(x,\omega)\in X\times Y^I\text{ such that }\omega(0) = \ast\text{ and }\omega(1) = f(x)\}.$$
Our description of $F(f,\ast)$ in \eqref{sethomotopyfiber} is almost exactly the same --- except that
the directions of the paths are reversed.
Thus there's a homeomorphism $F^\prime(f,\ast) \simeq F(f,\ast)$ given by reversing directions of paths.
\begin{remark}
    One could also replace both $f:X\to Y$ and $\ast\to Y$ by fibrations, and the resulting pullback is also homeomorphic
    to $F(f,\ast)$. (Prove this, if the statement is not immediate.)
\end{remark}

%Here's what you'll prove for homework.
%\begin{theorem}
%    Suppose you have two fibrations $p$ and $p^\prime$ such that the following diagram commutes, where $f$ is a homotopy equivalence.
%    \begin{equation*}
%	\xymatrix{
%	    E\ar[r]^p\ar[dr]^p & E^\prime\ar[d]^{p^\prime}\\
%	    & B
%	    }
%    \end{equation*}
%    Then $f$ is a fiber homotopy equivalence. That means that it's a homotopy equivalence in $\Top_{/B}$. What this means is that there is a map $g:E^\prime\to E$ over $B$ compatible with the fibrations and homotopies $I\times E\to E$ over $B$ and $I\times E^\prime\to E^\prime$ over $B$. I.e., the following three diagrams commute:
%    \begin{equation*}
%	\xymatrix{
%	    E^\prime\ar[r]^g\ar[dr]^{p^\prime} & E\ar[d]^p\\
%	    & B
%	    }
%    \end{equation*}
%    and 
%    \begin{equation*}
%	\xymatrix{
%	    I\times E\ar[r]^{1\sim gf}\ar[dr] & E\ar[d]^p\\
%	    & B
%	    }
%    \end{equation*}
%    and
%    \begin{equation*}
%	\xymatrix{
%	    I\times E^\prime\ar[r]^{1\sim fg}\ar[dr] & E^\prime\ar[d]^{p^\prime}\\
%	    & B
%	    }
%    \end{equation*}
%\end{theorem}
%So we find that for all $b$, $p^{-1}(b)\xrightarrow{\simeq} (p^\prime)^{-1}(b)$. So in particular, the fiber $F(f,\ast)$, i.e., the homotopy fiber, of $T(f)\to B$ and the fiber $f^{-1}(\ast)$ of $f:E\to B$ are homotopy equivalent if $f$ is a fibration.
