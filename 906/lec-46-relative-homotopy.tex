\section{Relative homotopy, $\pi_1$ action}
Pset 2 has question 8; there's still one more to go. Hood has office hours today 12 -- 1 in 2-390, and I have office hours today tomorrow 4-5 in 2-478.
\subsection{Spheres and homotopy groups}
Let's talk about spheres, first of all. The $n$-sphere is $S^n=I^n/\partial I^n$. Agreed? It's a standard choice. By the way, $S^1 = I/0\sim 1$. What about $S^0$? This is $\ast/\emptyset = \ast_+$. This is the usual way to think of the zero sphere. The point that you just added is the basepoint.

We have this looping thing going on; we know that $\Omega X$ (I'm always working in $\Top_\ast$ these days) is $X^{S^1}_\ast$. Thus the adjunction says that $\Top_\ast(W,\Omega X) = \Top_\ast(S^1\wedge W,X)$.
\begin{definition}
    The \emph{reduced suspension} $\Sigma W$ is $S^1\wedge W$.
\end{definition}
If I have $A\subseteq X$, then $X/A\wedge Y/B = (X\times Y)/((A\times Y)\cup_{A\times B}(X\times B))$. Thus $\Sigma X = S^1\wedge X$ is $I\times X/(\partial I \times X\cup I\times \ast)$. I collapse the top and bottom of a cylinder to a point, and also the line along a basepoint gets collapsed.

Similarly, $\Sigma^n X$ is the left adjoint of the $n$-fold loop space. Hence $\Sigma^n X = (S^1)^{\wedge n}\wedge X$. What is $S^1\wedge S^n$? This is $I/\partial I\wedge I^n\wedge \partial I^n = (I\times I^n)/(\partial I\times I^n\cup I\times \partial I^n)$. This denominator is exactly $\partial I^{n+1}$; hence $S^1\wedge S^n\simeq S^{1+n}$. In fact, $S^k\wedge S^n\simeq S^{k+n}$.
\begin{definition}
    The $n$th homotopy group of $X$ is $\pi_n X = \pi_0(\Omega^n X)$.
\end{definition}
This is $[S^0,\Omega^n X]_\ast = [S^n, X]_\ast = [(I^n,\partial I^n),(X,\ast)]$.
\subsection{The homotopy category}
We have the \emph{homotopy category of spaces} $\Ho(\Top)$ whose objects are spaces and whose mapping spaces are $\pi_0$ of mapping spaces. I have to check that if $f_0,f_1:X\to Y$ and $g:Y\to Z$, then $gf_0\simeq gf_1$. Of course, it is, just by composing with $g$. Similarly $f_0h\simeq f_1h$. This guarantees that you can define composition of maps in $\Ho(\Top)$. I can also think about $\Ho(\Top_\ast)$, with pointed homotopies. A lot of what we've been doing has been taking place in $\Ho(\Top_\ast)$.

Fix $W$. We need to check that $X\mapsto X^W_\ast$ is a homotopy functor. It defines a functor $\Top_\ast\to\Top_\ast$. Namely, I want to complete:
\begin{equation*}
    \xymatrix{
	\Top_\ast\ar[d]\ar[r]^{X\mapsto X^W_\ast} & \Top_\ast\ar[d]\\
	\Ho(\Top_\ast)\ar@{-->}[r] & \Ho(\Top_\ast)
    }
\end{equation*}
So I'd better check that if I have a homotopy $f_0\sim f_1:X\to Y$. Is that a map $I\wedge X\to Y$? This really tells you that there's a nullhomotopy if the basepoint of $I$ is one of the endpoints. I really want to consider $I\times X/I\times\ast$. Aha, but this is just $I_+\wedge X$. So a homotopy $f_0\sim f_1:X\to Y$ is a map $I_+\wedge X\to Y$.

Suppose I have a homotopy like this; then I get $(I_+\wedge X)^W\to Y^W_\ast$. I wanted $I_+\wedge X^W_\ast\to Y^W_\ast$, so that's not quite what I wanted. I can get this if I can construct a map $I_+\wedge X^W_\ast\to (I_+\wedge X)^W_\ast$. In fact, I want to construct $A\wedge X^W_\ast\to (A\wedge X)^W_\ast$. One thing I can do is $A\wedge X^W_\ast\to A^W_\ast\wedge X^W_\ast$ by sending $a\mapsto c_a$, and then the exponential law gives a homotop $A^W_\ast\wedge X^W_\ast\to (A\wedge X)^W_\ast$. This gives me a map $I_+\wedge X^W_\ast\to (I_+\wedge X)^W_\ast\to Y^W_\ast$ making $X\mapsto X^W_\ast$ a homotopy functor. In particular, $\Omega^n$ is a homotopy functor.

Let's continue with this homotopical localization of things.
\begin{definition}
    A fiber sequence in $\Ho(\Top_\ast)$ is a composite $X\to Y\to Z$ that is isomorphic in $\Ho(\Top_\ast)$ to some $Ff\xrightarrow{p} E\xrightarrow{f}B$. Namely we want some (possibly zig-zag of) maps that are homotopy equivalences:
    \begin{equation*}
	\xymatrix{
	    X\ar[r]\ar[d] & Y\ar[r]\ar[d] & Z\ar[d]\\
	    Ff\ar[r]_p & E\ar[r]_f & B
	    }
    \end{equation*}
\end{definition}
We've seen examples in our elaborate story. Recall our diagram:
\begin{equation*}
    \xymatrix{
	\cdots\ar[r] & Fp_4\ar[r] & Fp_3 \ar[r] & Fp_2\ar[r] & Fp_1\ar[r]^{p_2} & Ff\ar[r]^{p_1} & X\ar[r]^{f} & Y\\
    \cdots\ar[r] & \Omega Fp_1\ar[r]|{\overline{\Omega p_2}}\ar[u]_{\simeq} & \Omega Fp_0\ar[u]_{\simeq}\ar[ur]|{i(p_2)}\ar[r]|{\overline{\Omega p}} & \Omega X\ar[r]|{\overline{\Omega f}}\ar[u]_{\simeq}\ar[ur]|{i(p_1)} & \Omega Y\ar[u]_\simeq \ar[ur]|{i(p_0)} & &\\
	\Omega^2 X\ar[u]_{\simeq}\ar[r]_{\Omega f} & \Omega Y\ar[u]_{\simeq}\ar[ur]_{\overline{\Omega i(p_0)}} & & &
    }
\end{equation*}
So look, $Ff\to X\to Y$ is a fiber sequence. And $\Omega Y\to F\xrightarrow{p}X$ is another fiber sequence because it's isomorphic to $Fp\to F\to X$ in $\Ho(\Top_\ast)$. I also have $\Omega X\xrightarrow{\overline{\Omega f}}\Omega Y\to F$ is another fiber sequence. This means that $\Omega X\xrightarrow{\Omega f}\Omega Y\to F$ is another fiber sequence because these two fiber sequences differ by an automorphism of $\Omega X$ because in general, if $A^\prime\xrightarrow{\sim} A$ and $A\to B\to C$ is a fiber sequence, so is $A^\prime\xrightarrow{\sim} A\to B\to C$.

I can apply $\Omega$ again, so I get $\Omega F\xrightarrow{\Omega p} \Omega X\xrightarrow{\Omega f} \Omega Y$. I claim this is a fiber sequence, because this is a loop of a fiber sequence, and taking loops takes fiber sequences to fiber sequences. This is called the \emph{Barratt-Puppe sequence}. I carefully explained that this makes some sense, because it firstly makes sense to ask that $\Omega$ is a homotopy functor. I haven't said this yet. So in particular:
\begin{enumerate}
    \item $\Omega$ takes fiber sequences to fiber sequences.
    \item $\Omega Ff\simeq F\Omega f$. Check this!
\end{enumerate}
You can now \emph{loop back} to get $\Omega^2 Y\xrightarrow{\Omega i} \Omega F\xrightarrow{\Omega p}\Omega X$. This is an unstable version of a triangulated category. It wants to be a triangulated category, but it isn't.
\begin{remark}
    If $f:X\to Y$, I can form $Ff$. I might have a homotopy commuting diagram like:
    \begin{equation*}
	\xymatrix{
	    \Omega Y\ar[d]_{\Omega g}\ar[r] & Ff\ar@{-->}[d]\ar[r] & X\ar[d]_{h}\ar[r]^f & Y\ar[d]^g\\
	    \Omega Y^\prime\ar[r] & Ff^\prime\ar[r] & X^\prime\ar[r]_{f^\prime} & Y
	    }
    \end{equation*}
    The dotted map exists, but \emph{this map depends on the homotopy} $f^\prime h\simeq gf$. That's an important subtlety.
\end{remark}
\subsection{Lexseq of a fiber sequence}
Applying $\pi_0 = [S^0,-]_\ast$ to the Barratt-Puppe sequence gives a lexseq:
\begin{equation*}
    \xymatrix{
	& \cdots\ar[r] & \pi_2 Y\ar[dll]\\
	\pi_1 F\ar[r] & \pi_1 X\ar[r] & \pi_1 Y\ar[dll]\\
	\pi_0 F\ar[r] & \pi_0 X\ar[r] & \pi_1 X
    }
\end{equation*}
of pointed sets. But this is even better, ,because $\Omega X$ is a group object in $\Ho(\Top_\ast)$.
\begin{remark}
    $\Ho(\Top)$ and $\Ho(\Top_\ast)$ has products and coproducts, but very few other limits or colimits. So as a category, it is \emph{horrible}.
\end{remark}
I showed you an argument that $\Omega^2 X$ is an \emph{abelian} group object, i.e., multiplication is commutative up to homotopy. Thus $\pi_1$ is a group and $\pi_k$ is an abelian group for $k\geq 2$; hence in our diagram above, all maps (except on $\pi_0$) are group homomorphisms!

What if $X\to Y$ is the inclusion $i:A\hookrightarrow X$ of a subspace? Then $Fi=\{(a,\omega)\in A\times X^I_\ast|\omega(1) = a\}$. This is just the collection of all paths that begin at $\ast\in A$ and ends in $A$.
\begin{definition}
    $\pi_n(X,A,\ast) = \pi_n(X,A)$ is $\pi_{n-1}Fi = [(I^n,\partial I^n,(\partial I^n\times I)\cup (I^{n-1}\times 0)),(X,A,\ast)]$.
\end{definition}
Inside $I^n$ is $\partial I^n$, and also including in it is $\partial I^n\times I\cup I^{n-1}\times 0$. Then you can check that $\pi_{n-1}Fi = [(I^n,\partial I^n,(\partial I^n\times I)\cup (I^{n-1}\times 0)),(X,A,\ast)]$. In this case, you have a lexseq: 
\begin{equation*}
    \xymatrix{
	& \cdots\ar[r] & \pi_2 (X,a)\ar[dll]\\
	\pi_1 A\ar[r] & \pi_1 X\ar[r] & \pi_1 (X,A)\ar[dll]\\
	\pi_0 A\ar[r] & \pi_0 X\ar[r] & 
    }
\end{equation*}
