\section{Cellular approximation, cellular homology, obstruction theory}
\todo{edit from here}
pset 3 is up in part. Today I'm going to tell you how to think about CW-complexes and work with them, and I won't give a lot of details. Some good sources are lecture notes by Davis-Kirk, and a beautiful book by Glen Bredon, as well as Hatcher.
\subsection{Cellular approximation}
\begin{theorem}
    Let $X$ and $Y$ be CW-complexes and $A\subseteq X$ a subcomplex. Let $f:X\to Y$ be a continuous map, and that $f|_{A}$ is skeletal\footnote{Some people would say cellular.}, i.e., $f(\Sigma_n)\subseteq Y_n$. It might not take cells in $A$ to cells in $Y$, but it takes $n$-skeleta to $n$-skeleta. Then $f$ is homotopic to some other $f^\prime:X\to Y$ relative to $A$ with $f^\prime$ skeletal on all of $X$.
\end{theorem}
This is kind of amazing.
\begin{lemma}[Key lemma]
    Any $(D^n,S^{n-1})\to (Y,Y_{n-1})$ compresses through:
    \begin{equation*}
	\xymatrix{
	(D^n,S^{n-1})\ar[r]\ar@{-->}[dr] & (Y,Y_{n-1})\\
	& (Y_n,Y_{n-1})\ar[u]
	}
    \end{equation*}
\end{lemma}
\begin{proof}[``Proof.'']
    I take this as obvious, but here's something. Since $D^n$ is compact, we know that $f(D^n)$ must lie in some finite subcomplex $K$ of $Y$. Now the map $D^n\to K$ might hit some top-dimensional cell $e^m\subseteq K$, which doesn't have anything attached to it, so I can poke a hole in it (namely the approximate this so that you miss a point) and it contracts onto a lower-dimensional cell. Iterating this process gives the desired result.
\end{proof}
Let's see why the theorem is true based on this lemma.
\begin{proof}[``Proof'' of the cellular approximation theorem]
    I'm going to make this homotopy $f\simeq f^\prime$ one cell at a time. I might as well replace $A$ by whatever I've extended the homotopy to. So I have a single cell attachment $A\to A\cup D^m$. We have:
    \begin{equation*}
	\xymatrix{
	A\ar[d]_{\text{skeletal}}\ar[r] & A\cup D^m\ar[dl]^{\text{may not be skeletal}}\\
	Y &
	}
    \end{equation*}
    Now we're to use the compression lemma to get $A\cup D^m\to Y_m$. I'm not satisfied, because I have to do something about the rest of $X$. Here's a wonderful fact: the inclusion of a subcomplex is a cofibration. Thus we can extend this to a map from $X$, and iterate to get what we want.

    Why's the wonderful fact true? We have a cofibration $S^{n-1}\to D^n$, so any coproduct of these is a cofibration, and the pushout of a cofibration is a cofibration.
\end{proof}
\begin{corollary}[Homework]
    The pair $(X,X_n)$ is $n$-connected.
\end{corollary}
\subsection{Cellular homology}
Let $(X,A)$ be a relative CW-complex with $A\subseteq X_{n-1}\subseteq X_n\subseteq\cdots\subseteq X$. We showed last fall that $H_\ast(X_n,X_{n-1})\simeq \widetilde{H}_\ast(X_n/X_{n-1})$. More generally, if $B\to Y$ is a cofibration, then $H_\ast(Y,B)\simeq\widetilde{H}_\ast(Y/B)$. You can find a precise proof in Bredon, page 433. Anyway, $X_n/X_{n-1} = \bigvee_{\alpha\in \Sigma_n}S^n_\alpha$, so that $H_\ast(X_n,X_{n-1})\simeq \Z[\Sigma_n] = C_n(X,A)$. The composite $S^{n-1}\to X_{n-1}\to X_{n-1}/X_{n-2}$ is called a relative attaching map.

There's a boundary map $d:C_n = H_n(X_n,X_{n-1})\xrightarrow{\partial}H_{n-1}(X_{n-1})\to H_{n-1}(X_{n-1},X_{n-2}) = C_{n-1}$. You can check that $d^2 = 0$. The beautiful theorem is that $H_n(X,A)\simeq H_n(C_\ast(X,A))$. We showed this last fall, but not for relative CW-complexes. Cellular approximation tells us that you can compute the effect of maps on homology.

I could do the same thing of course for cohomology, to get $C^n(X,A;\pi) = \Hom(C_n(X,A),\pi) = \Map(\Sigma_n,\pi)$. Here $\pi$ is any abelian group. I use $\pi$ because it's gonna be a homotopy group in just a minute.
\subsection{Obstruction theory}
The question is: can we extend:
\begin{equation*}
    \xymatrix{
	X\ar@{-->}[dr] & \\
	A\ar@{^(->}[u]\ar[r]_f & Y
    }
\end{equation*}
We can work out the lower level obstructions
\begin{equation*}
    \xymatrix{
	A\ar[d]\ar@{^(->}[r] & X_0\ar@{-->}[dl]\ar@{^(->}[r] & X_1\ar@{-->}[dll]\\
	\emptyset\neq Y & &
    }
\end{equation*}
for instance if two points in $X_0$ get connected in $X_1$ we've to check that they're connected in $Y$.

For $n\geq 2$, we can form the diagram:
\begin{equation*}
    \xymatrix{
	\coprod_{\alpha\in\Sigma_n}S^{n-1}_\alpha\ar[r]^f\ar@{^(->}[d] & X_{n-1}\ar[r]^g\ar[d] & Y\\
	\coprod_{\Sigma_n}D^n_\alpha\ar[r] & X_n\ar@{-->}[ur] & 
    }
\end{equation*}
If the composite $S^{n-1}_\alpha\to X_{n-1}\to Y$ is nullhomotopic, the extension exists.

We find that $g\circ f_\alpha\in [S^{n-1},Y]$. Let's assume that $Y$ is simple (it's path connected and the action of $\pi_1$ is trivial, so all homotopy groups are abelian). Then $[S^{n-1},Y] = \pi_{n-1}(Y)$. I've given you a map $\Sigma_n\xrightarrow{\theta}\pi_{n-1}(Y)$, which is now a cochain in $C^n(X,A;\pi_{n-1}(Y))$, and it has the property that $\theta = 0$ if and only if $g$ extends to $X_n\to Y$.
\begin{prop}
    $\theta$ is a cocycle, called the ``obstruction cocycle''.
\end{prop}
\begin{proof}
    $\theta$ gives a map $H_n(X_n,X_{n-1})\to \pi_{n-1}(Y)$, and I want to see that this composite $H_{n+1}(X_{n+1},X_n)\xrightarrow{\partial}H_n(X_n)\to H_n(X_n,X_{n-1})\xrightarrow{\theta}\pi_{n-1}(Y)$ is zero. OK, we know that $\pi_n(X_n,X_{n-1})\to H_n(X_n,X_{n-1})$ is surjective. This relative homotopy group holds the characteristic maps, doesn't it? So picking a representative back there is no problem; now I have the lexseq of a pair, which gives:
    \begin{equation*}
	\xymatrix{
	    \pi_{n+1}(X_{n+1},X_n)\ar[d]\ar@{->>}[r] & H_{n+1}(X_{n+1},X_n)\ar[d]^\partial\\
	    \pi_n(X_n)\ar[r]\ar[d] & H_n(X_n)\ar[d]\\
	    \pi_n(X_n,X_{n-1})\ar@{->>}[r] \ar[d]_\partial & H_n(X_n,X_{n-1})\ar[d]^\theta\\
	    \pi_{n-1}(X_{n-1})\ar[r]_{g_\ast} & \pi_{n-1}(Y)
	    }
    \end{equation*}
    This diagram commutes, so $\theta$ is indeed a cocycle.
\end{proof}
\begin{theorem}
    Let $(X,A)$ be a CW-complex and $Y$ a simple space. Suppose I have $g:X_{n-1}\to Y$. Then $g|_{X_{n-2}}$ extends to $X_n$ iff $[\theta(g)]\in H^n(X,A;\pi_{n-1}(Y))$ is zero.
\end{theorem}
This is a dream come true because it's a cohomological condition for extensions!
\begin{corollary}
    If $H^n(X,A;\pi_{n-1}(Y)) = 0$ for all $n>2$, then any:
    \begin{equation*}
	\xymatrix{
	A\ar[r]\ar[d] & Y\\
	X\ar@{-->}[ur] & 
	}
    \end{equation*}
    extends up to homotopy (and thus strictly because $A\to X$ is a cofibration).
\end{corollary}
\begin{example}
    If $X = CA$, does every map $A\to Y$ factor through the cone? The condition is that $H^n(CA,A;\pi_{n-1}(Y)) \simeq \widetilde{H}^{n-1}(A;\pi_{n-1}(Y)) = 0$.
\end{example}
It doesn't make it any easier to compute what the homotopy groups of $Y$ are, but often times you can obtain some information.
\begin{remark}
    Fix a prime $p$. Later we'll see (I hope) that $Y$ is simply connected (maybe even simple?), and suppose that $\widetilde{H}_\ast(Y;\Z_{(p)}) = 0$ (i.e., no $\Z$-summands and no $\Z/p^k$-summands). Then $\pi_\ast(Y)\otimes\Z_{(p)} = 0$ too. So if $H_\ast(A;\Z/\ell\Z) = 0$ for $\ell\neq p$, then $[A,Y]\simeq \ast$.
\end{remark}
