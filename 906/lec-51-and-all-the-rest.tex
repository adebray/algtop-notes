\section{Conclusions from obstruction theory}
The main result of obstruction theory, as discussed in the previous section, is the following.
\begin{theorem}[Obstruction theory]
    Let $(X,A)$ be a relative CW-complex, and $Y$ a simple space. The map $[X,Y]\to [A,Y]$ is:
    \begin{enumerate}
	\item is onto if $H^n(X,A;\pi_{n-1}(Y)) = 0$ for all $n\geq 2$.
	\item is one-to-one if $H^n(X,A;\pi_n(Y)) = 0$ for all $n\geq 1$.
    \end{enumerate}
\end{theorem}
\begin{remark}
    The first statement implies the second.
    Indeed, suppose we have two maps $g_0,g_1:X\to Y$ and a homotopy $h:g_0|_{A}\simeq g_0|_{A}$.
    Assume the first statement.
    Consider the relative CW-complex $(X\times I,A\times I\cup X\times\partial I)$.
    Because $(X,A)$ is a relative CW-complex, the map $A\hookrightarrow X$ is a cofibration; this implies that the map $A\times I\cup X\times\partial I\to X\times I$ is also a cofibration.
    $$H^n(X\times I,A\times I\cup X\times\partial I;\pi)\simeq \widetilde{H}^n(X\times I/(A\times I\cup X\times\partial I);\pi) = H^n(\Sigma X/A;\pi)\simeq \widetilde{H}^{n-1}(X/A;\pi).$$
%More precisely, if I have $X_{n-1}\to Y$, we get $\theta\in Z^n(X,A;\pi_{n-1}(Y))$, constructed by looking at the attaching map $f_\alpha$ of some $\alpha\in\Sigma_n$, to define $\theta(g)$ via $\theta(g)(\alpha) = [g\circ f_\alpha]$. This captures the obstruction to extending $g$ over $\alpha$. We found that $d\theta = 0$.
\end{remark}
We proved the following statement in the previous section.
\begin{prop}
    Suppose $g:X_{n-1}\to Y$ is a map from the $(n-1)$-skeleton of $X$ to $Y$. Then $g|_{X_{n-2}}$ extends to $X_n\to Y$ iff $[\theta(g)] = 0$ in $H^n(X,A;\pi_{n-1}(Y))$.
\end{prop}
An immediate consequence is the following.
\begin{theorem}[CW-approximation]
    Any space admits a weak equivalence from a CW-complex.
\end{theorem}
This tells us that studying CW-complexes is not very restrictive, if we work up to weak equivalence.

It is easy to see that if $W$ is a CW-complex and $f:X\to Y$ is a weak equivalence, then $[W,X]\xrightarrow{\simeq}[W,Y]$. 
We can now finally conclude the result of Theorem \ref{weakhtpyequiv}:
\begin{corollary}
    Let $X$ and $Y$ be CW-complexes.
    Then a weak equivalence $f:X\to Y$ is a homotopy equivalence.
\end{corollary}
\subsection{Postnikov and Whitehead towers}
Let $X$ be path connected.
There is a space $X_{\leq n}$, and a map $X\to X_{\leq n}$ such that $\pi_i(X_{\geq n}) = 0$ for $i>n$,
and $\pi_i(X)\xrightarrow{\simeq}\pi_i(X_{\leq n})$ for $i\leq n$.
This pair $(X,X_{\leq n})$ is essentially unique up to homotopy; the space $X_{\leq n}$ is called 
the \emph{$n$th Postnikov section} of $X$.
Since Postnikov sections have ``simpler'' homotopy groups, we can try to understand $X$ by studying each of its Postnikov
sections individually, and then gluing all the data together.

Suppose $A$ is some abelian group.
We saw, in the first part\todo{provide a link} that there is a space $M(A,n)$ with homology given by:
\begin{equation*}
    \widetilde{H}_i(M(A,n)) = \begin{cases}
	A & i = n\\
	0 & i\neq n.
    \end{cases}
\end{equation*}
This space was constructed from a free resolution $0\to F_1\to F_0\to A\to 0$ of $A$.
We can construct a map $\bigvee S^n\to \bigvee S^n$ which realizes the first two maps; coning this off gets $M(A,n)$.
By Hurewicz, we have:
\begin{equation*}
    \pi_i(M(A,n)) = \begin{cases}
	0 & i<n\\
	A & i = n\\
	?? & i>n
    \end{cases}
\end{equation*}
It follows that, when we look at the $n$th Postnikov section of $M(A,n)$, we have:
\begin{equation*}
    \pi_i(M(A,n)_{\leq n}) = \begin{cases}
	A & i = n\\
	0 & i\neq n.
    \end{cases}
\end{equation*}
In some sense, therefore, this Postnikov section is a ``designer homotopy type''.
It deserves a special name: $M(A,n)_{\leq n}$ is called an \emph{Eilenberg-MacLane space}, and is denoted $K(A,n)$.
By the fiber sequence $\Omega X\to PX\to X$ with $PX\simeq \ast$, we find that $\Omega K(\pi,n)\simeq K(\pi,n-1)$.
Eilenberg-MacLane spaces are unique up to homotopy.

Note that $n=1$, $A$ does not have to be abelian, but you can still construct $K(A,1)$.
This is called the \emph{classifying space} of $G$; such spaces will be discussed in more detail in the next chapter.
Examples are in abundance: if $\Sigma$ is a closed surface that is not $S^2$ or $\RR^2$, then $\Sigma \simeq K(\pi_1(\Sigma),1)$. 
Perhaps simpler is the identification $S^1\simeq K(\Z,1)$.

\begin{example}
    We can identify $K(\Z,2)$ as $\CP^\infty$.
    To see this, observe that we have a fiber sequence $S^1\to S^{2n+1}\to \CP^n$.
    The long exact sequence in homotopy tells us that the homotopy groups of $\CP^n$ are the same as the homotopy groups of $S^1$,
    until $\pi_\ast S^{2n+1}$ starts to interfere.
    As $n$ grows, we obtain a fibration $S^1\to S^\infty\to \CP^\infty$.
    Since $S^\infty$ is weakly contractible (it has no nonzero homotopy groups), we get the desired result.
\end{example}
\begin{example}
    Similarly, we can identify $K(\Z/2\Z,1)$ as $\RP^\infty$.
\end{example}

Since $\pi_1(K(A,n)) = 0$ for $n>1$, it follows that $K(A,n)$ is automatically a simple space.
This means that
$$[S^k,K(A,n)] = \pi_k(K(A,n)) = H^n(S^k,A).$$
In fact, a more general result is true:
\begin{theorem}[Brown representability]
    If $X$ is a CW-complex, then $[X,K(A,n)] = H^n(X;A)$. 
\end{theorem}
We will not prove this here, but one can show this simply by showing that the functor $[-,K(A,n)]$ satisfies
the Eilenberg-Steenrod axioms.
Somehow, these Eilenberg-MacLane spaces $K(A,n)$ completely capture cohomology in dimension $n$. 

If $X$ is a CW-complex, then we may assume that $X_{\leq n}$ is also a CW-complex.
(Otherwise, we can use cellular approximation and then kill homotopy groups.)
Let us assume that $X$ is path connected; then $X_{\leq 1} = K(\pi_1(X),1)$.
We may then form a (commuting) tower:
\begin{equation*}
    \xymatrix{
	& \vdots\ar[d] & \cdots\ar[l]\\
	& X_{\leq 3}\ar[d]& K(\pi_3(X),3)\ar[l]\\
	& X_{\leq 2}\ar[d]& K(\pi_2(X),2)\ar[l]\\
	X\ar[r]\ar[ur]\ar[uur]\ar[uuur]& X_{\leq 1}\ar@{=}[r] & K(\pi_1(X),1),
    }
\end{equation*}
since $K(\pi_n(X),n)\to X_{\leq n}\to X_{\leq n-1}$ is a fiber sequence.
This decomposition of $X$ is called the \emph{Postnikov tower} of $X$.

Denote by $X_{>n}$ the fiber of the map $X\to X_{\leq n}$ (for instance, $X_{>1}$ is the universal cover of $X$); then, we have
\begin{equation*}
    \xymatrix{
	\cdots\ar[r]\ar[d] & \cdots\ar[r]\ar@{=}[d] & \vdots\ar[d] & \cdots\ar[l]\\
	X_{>3}\ar[r]\ar[d] & X\ar[r]\ar@{=}[d] & X_{\leq 3}\ar[d]& K(\pi_3(X),3)\ar[l]\\
	X_{>2}\ar[r]\ar[d] & X\ar[r]\ar@{=}[d] & X_{\leq 2}\ar[d]& K(\pi_2(X),2)\ar[l]\\
	X_{>1}\ar[r]\ar[d] & X\ar[r]\ar@{=}[d] & X_{\leq 1}\ar@{=}[r]\ar[d] & K(\pi_1(X),1)\\
	X\ar@{=}[r] & X\ar[r] & \ast
    }
\end{equation*}
The leftmost tower is called the \emph{Whitehead tower} of $X$, named after George Whitehead.

I can take the fiber of $X_{>1}\to X$, and I get $K(\pi_1(X),0)$;
more generally, the fiber of $X_{>n} \to X_{>n-1}$ is $K(\pi_n(X),n-1)$.
This yields the following diagram:
\begin{equation*}
    \xymatrix{
	\cdots & \vdots\ar[d] & \vdots\ar@{=}[d] & \vdots\ar[d] & \cdots\\
	K(\pi_3(X),2)\ar[r] & X_{>3}\ar[r]\ar[d] & X\ar[r]\ar@{=}[d] & X_{\leq 3}\ar[d]& K(\pi_3(X),3)\ar[l]\\
	K(\pi_2(X),1)\ar[r] & X_{>2}\ar[r]\ar[d] & X\ar[r]\ar@{=}[d] & X_{\leq 2}\ar[d]& K(\pi_2(X),2)\ar[l]\\
	K(\pi_1(X),0)\ar[r] & X_{>1}\ar[r]\ar[d] & X\ar[r]\ar@{=}[d] & X_{\leq 1}\ar@{=}[r]\ar[d] & K(\pi_1(X),1)\\
	& X\ar@{=}[r] & X\ar[r] & \ast
    }
\end{equation*}

We can construct Eilenberg-MacLane spaces as cellular complexes by attaching cells to the sphere to kill its higher homotopy groups.
The complexity of homotopy groups, though, shows us that attaching cells to compute the cohomology of Eilenberg-MacLane spaces
is not feasible.
%These constructions go back to the 50's, and they had voluminous computations in low dimensions. One day in 1950, they got a postcard from Serre, who said, ``here's a computation you might be interested in: $H^{23}(K(\Z,14)) = ...$''. Of course, Serre and Cartan had a different approach, that was much more effective. They observed that the fact $\Omega K(\pi,n)\simeq K(\pi,n-1)$ wasn't perceived to be useful by Eilenberg and Maclane. They didn't think about fiber sequences. Serre and Cartan did this by means of a spectral sequence. We'll do that later in the course.
