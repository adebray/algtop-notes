\section{Principal bundles, associated bundles}
Since Tuesday didn't exist, I'll have office hours on Thursday from 4 to 5 in 2-478.
\begin{definition}
    Suppose $E,E^\prime\to B$ are vector bundles over $B$. An \emph{isomorphism} is a map $\alpha:E\to E^\prime$ over $b$ that's a linear isomorphism on each fiber.
\end{definition}
This implies it's invertible.
\begin{notation}
    Denote by $\Vect(B)$ the set of isomorphism classes of vector bundles over $B$.
\end{notation}
Why's that a set? I'll let you think about that.

Suppose $\xi\downarrow B$. If $f:B^\prime\to B$, taking the pullback gives a vector bundle denoted $f^\ast\xi$. This operation descends to a map $f^\ast:\Vect(B)\to \Vect(B^\prime)$. This operation certainly is functorial, and we get $\Vect:\Top^{op}\to \Set$.
\begin{warning}
In order to say anything meaningful about vector bundles, you've to assume that the vector bundles are numerable. If $B$ is paracompact (eg a CW-complex), this is automatic.
\end{warning}
We'd like to understand this functor $\Vect$.
\begin{theorem}
    $\Vect$ is $I$-invariant (here $I = \Delta^1$), i.e. the projection $X\times I\to X$ induces an isomorphism $\Vect(X)\to \Vect(X\times I)$.
\end{theorem}
This is a basic theorem, and we'll prove this. Let me point out a corollary of this.
\begin{corollary}
    $\Vect$ is a homotopy functor.
\end{corollary}
\begin{proof}
    Suppose $f,g:B\to B^\prime$ with $f\simeq g$. Then I have $H:B^\prime\times I\to B$. If $\xi\downarrow B$, I'd like to see that $f^\ast_0\xi\simeq f_1^\ast\xi$. This is far from obvious. Well, we have:
    \begin{equation*}
	\xymatrix{
	    B^\prime\times I\ar[r]\ar[d]_{pr} & B\\
	    B^\prime & 
	    }
    \end{equation*}
    The left map is an isomorphism under $\Vect$. Suppose $\eta\downarrow B$ such that $pr^\ast\eta \simeq f^\ast\xi$.
    For any $t\in I$, we know that:
    $$f_t^\ast\xi \simeq in_t^\ast f^\ast\xi \simeq in_t^\ast pr^\ast\eta \simeq (pr\circ in_t)^\ast\eta \simeq \eta$$
    where $in_t:B^\prime\to B^\prime\times I$ sends $x\mapsto(x,t)$. That's the proof.
\end{proof}
OK, note that $\Vect(X)\to \Vect(X\times I)$ being injective is obvious. I'll prove surjectivity on Friday. I'll prove something more general, in fact.
\subsection{Principal bundles}
There's a famous video of J.-P.~Serre talking about mathematics. He says you have to know the difference between ``principle'' and ``principal''. He contemplated ``bundles of principles'' (like politics, or society, or something).
\begin{definition}
    Let $G$ be a topological group\footnote{Really, I only care about discrete groups and Lie groups.}. A \emph{principal $G$-bundle} is a right action of $G$ on $P$ such that:
    \begin{itemize}
	\item $G$ acts freely.
	\item The orbit projection $P\to P/G$ is a fiber bundle.
    \end{itemize}
\end{definition}
\begin{example}
    Suppose $G$ is discrete. The fibers are discrete, so the condition that $P\to P/G$ is a fiber bundle \emph{is} that it's a covering projection, i.e., the action is ``properly discontinuous''.
    As a special case, suppose $X$ is a space with universal cover $\widetilde{X}\downarrow X$. Then $\pi_1(X)$ acts freely on $\widetilde{X}$, and $\widetilde{X}\downarrow X$ is the orbit projection.
    So this is a principal bundle; for instance, $S^{n-1}\downarrow\RP^{n-1}$ is a principal $\Z/2\Z$-bundle. The Hopf fibration $S^{2n-1}\downarrow \CP^{n-1}$ is a principle $S^1$-bundle.
\end{example}
So this is \emph{not} an unfamiliar object.

By looking at the universal cover, we can classify covering spaces of $X$. Remember how that goes: let $F$ be a set with left $\pi_1(X)$-action. Then you have the dotted map, which is the desired covering space:
\begin{equation*}
    \xymatrix{
	\widetilde{X}\times F\ar[r]\ar[d]_{p\circ pr_1} & \widetilde{X}\times F/\sim\ar[dl]^q\\
	X & 
    }
\end{equation*}
where $(y,gz)\sim (yg,z)$ where $y\in\widetilde{X}$, $z\in F$, and $g\in\pi_1(X)$.

Fix $y_0\in\widetilde{X}$ over $\ast\in X$, I claim that $F\xrightarrow{\simeq}q^{-1}(\ast)$ via $z\mapsto (y_0,z)$.
I'll let you check that that's a bijection -- that's supposed to be clear.
\begin{theorem}[Covering spaces]
    There's an equivalence of categories:
    $$\{\text{Left $\pi_1(X)$-sets}\}\xrightarrow{\simeq}\{\text{Covering spaces of }X\}$$
    with inverse functor given by taking the fiber over the basepoint and lifting a loop in $X$ to get a map from the fiber to itself.
\end{theorem}
How about a more general picture?
\begin{construction}
If $P\downarrow B$ is a principal $G$-bundle, and $F$ is a left $G$-space, we get a new fiber bundle via:
\begin{equation*}
    \xymatrix{
	P\times F\ar[r]\ar[d] & P\times F/\sim\ar[dl]^q\\
	B & 
    }
\end{equation*}
    This gives a new fiber bundle with fiber $F$, for the following reason. Let $x\in B$, and let $y\in P$ over $x$. We get $F\xrightarrow{\simeq} q^{-1}(\ast)$ via $z\mapsto[y,z]$. Define $q^{-1}(\ast)\to F$ via $[y^\prime,z^\prime]=[y,gz^\prime]\mapsto gz^\prime$ where $y^\prime = yg$ for some unique $g$. You can check that these two maps are inverse homeomorphisms.
    This is called an ``associated bundle'', and it's denoted $P\times_G F$.
\end{construction}
Let $\xi\downarrow B$ be an $n$-plane bundle (all the fiber dimensions are the same, say $n$). We can construct a principal $\GL_n(\RR)$-bundle $P(\xi)$. Define $P(\xi)_b = \text{set of bases for }E(\xi)_b = \mathrm{Iso}(\RR^n, E(\xi)_b)$. What's the topology?
Well, $P(B\times \RR^n) = B\times \mathrm{Iso}(\RR^n,\RR^n)$, topologically, where $\mathrm{Iso}(\RR^n,\RR^n) = \GL_n(\RR)$ is given the usual topology as a subspace of $\RR^{n^2}$.
There's an obvious right action of $\GL_n(\RR)$ on $P(\xi)\downarrow B$, given by precomposition. Obviously the action is free, and it's simply transitive, so you actually have a \emph{principal action} of $\GL_n(\RR)$ on $P(\xi)$. This is called the \emph{principalization} of $\xi$. 

Look at the associated bundle with fiber $F = \RR^n$, where $\GL_n(\RR)$ acts on $\RR^n$ from the left, as it usually does. So I can form $P(\xi)\times_{\GL_n(\RR)}\RR^n$. Because this is a linear action, it's a vector bundle, and $P(\xi)\times_{\GL_n(\RR)}\RR^n\simeq E(\xi)$.

Fix a topological group $G$. Define $\Bun_G(B)$ as the set of isomorphism classes of $G$-bundles over $B$. An isomorphism is a $G$-equivariant homeomorphism over the base.
We've established that there's a natural isomorphism of functors:
$$\Bun_{\GL_n(\RR)}(B) \simeq \Vect(B)$$
The $I$-invariance theorem will follow immediately from:
\begin{theorem}
    $\Bun_G$ is $I$-invariant.
\end{theorem}
There's a lot more that I wanted to say, but let me say one additional thing.

Principal bundles allow a description of geometric structures on $\xi$. By that I mean: suppose for instance that I have a metric on $\xi$. Then instead of all ordered bases, I can look at all ordered orthonormal bases in each fiber. This give the \emph{frame bundle} $\mathrm{Fr}(B) = \text{ordered orthonormal bases of }E(\xi)_b$, i.e., isometric isomorphisms $\RR^n\to E(\xi)_b$. Again, I have an action of the orthogonal group on $\mathrm{Fr}(B)$, so you have a principal $O(n)$-bundle. More examples: consistent orientations give an $SO(n)$-bundle. Trivializations give principal bundles as well. This is called ``reduction of the structure group''.
