\section{Bundles, classifying spaces, simplicial objects}
pset 4 is complete; it's due April 12.

I learned this attitude towards classifying spaces from Segal.
An article is on the website.

A bizarre thing we tried to do was look at the category of categories.
Among these are groups.
We took the nerve of a category, and got a simplicial set.
Then we took the geometric realization.
All in all, we have:
\begin{equation*}
    \xymatrix{
	\Cat\ar[r]^{\text{nerve}} & s\Set\ar[d]^{|-|}\\
	\mathbf{Gp}\ar[u]\ar[u]_{B} & \Top
    }
\end{equation*}
$BG$ for a group $G$ is called the classifying space, although it isn't clear what it's classifying at this point.

Another example of a category is given by $[n]$.
The classifying space of $[n]$ is canonically isomorphic to $\Delta^n$, as we observed.
You will discover for homework that $B$ takes products to products.
This leads to the fact that the classifying space construction sends natural transformations to homotopies.
This is a sort of 2-categorical thing.
\begin{lemma}
    Let $G$ be a group, and $g\in G$.
    Let $c_g:G\to G$ via $x\mapsto gxg^{-1}$.
    This gives a self map $Bc_g:BG\to BG$.
    This map is homotopic to the identity.
\end{lemma}
\begin{proof}
    The homomorphism $c_g$ is a functor from $G$ to itself.
    The claim is that there is a natural transformation from the identity to that functor.
    What is this natural transformation, call it $\theta:1\to c_g$?
    It sends the only object to the only object.
    Define $\ast\to \ast$ via $\ast\xrightarrow{g}\ast$.
    We then need the following diagram to commute, which it obviously does:
    \begin{equation*}
	\xymatrix{
	    \ast\ar[d]_1\ar[r]^g & \ast\ar[d]^{gxg^{-1}}\\
	    \ast\ar[r]_{g} & \ast
	    }
    \end{equation*}
    This is one of my favorite proofs.
\end{proof}
Groups are born to act.
Let's let $G$ act on some set $X$.
More generally, define an action of a category $\cc$ as a functor $\cc\xrightarrow{X}\Set$.
\begin{definition}
    The ``translation'' category $X\cc$ has $\mathrm{ob}(X\cc) = \coprod_{c\in \cc}X_c$, and morphisms defined via $X\cc(x\in X_c, y\in X_d) = \{f:c\to d:f_\ast(x) = y\}$.
\end{definition}
There is a projection $X\cc\to \cc$.
This is a special case of the Grothendieck construction.
\begin{example}
    Suppose $G$ acts on itself by left translation.
    Let $\widetilde{G}$ for this $G$-set.
    Then $\widetilde{G}G$ has objects as $G$, and maps $x\to y$ are elements $yx^{-1}$.
    This category is ``unicursal''.
    There's exactly one map from one object to another object.
    Every object is therefore initial and terminal, so the classifying space of this category is trivial!
    I.e., $B(\widetilde{G}G)\simeq\ast$.
    This is denoted $EG$.
    It comes equipped with a map down to $BG$, because we have a map $\widetilde{G}G\to G$.
\end{example}
You could also have $G$ acting by right translation.
The right and left action commute with each other, by associativity.
Thus the right action is equivariant with respect to the left action.
Consequently, we get a right action of $G$ on $EG$.

Here's the claim: this action is a principal action, and the orbit projection is $EG\to BG$.
I've produced a topologically interesting thing.
Note that $EG$ is a CW-complex -- the classifying space is always a CW-complex!

Let's think about $N(\widetilde{G}G)_n$.
An element is a chain of composable morphisms -- it's actually just a sequence of $n+1$ elements in the group.
I.e., $N(\widetilde{G}G)_n = G^{n+1}$.
The right action of the group on this is the diagonal action.
\begin{lemma}
    This is a free action.
    More precisely, if $G$ is a group and $X$ is a $G$-set, and if $X\times^\Delta G$ has the diagonal $G$-action and $X\times G$ has $G$ acting on the second factor by right translation, then $X\times^\Delta G\simeq X\times G$ as $G$-sets.
\end{lemma}
The word for this is ``shearing''.
\begin{proof}
    Define $X\times^\Delta G\mapsto X\times G$ via $(x,g)\mapsto (xg^{-1},g)$.
    It's clearly bijective.
    Is this equivariant?
    Well, $(x,g)\cdot h = (xh, gh)$, while $(xg^{-1}, g) \cdot h = (xg^{-1}, gh)$.
    The element $(xh, gh)$ is sent to $(xh(gh)^{-1}, gh)$, and so we're done.
    
    For good measure, define $X\times G\to X\times^\Delta G$ via $(x,g)\mapsto (xg, g)$.
\end{proof}
Isn't that cool?

OK, so $G$ acts freely on $N(\widetilde{G}G)_n$.
So a nonidentity group element is always going to send a simplex to another simplex.
I.e., $G$ acts freely on $EG$.

What's the orbit space?
If I take that diagonal $G$-action, when I divide by $G$, according to the shear isomorphism, I'm just canceling out one copy of $G$.
I.e., $N(\widetilde{G} G)/G\simeq G^n\simeq (NG)_n$.
Of course, you still have to check degeneracies and face maps to see if they work out.
But they do, and the realization is $BG$.

Everything I said here works just as well if we were to consider functors $X:\cc\to \Top$ instead of $X:\cc\to \Set$.
I want to study $G$ when it's a topological group, like Lie groups.
Define $\widetilde{G}$ to be a category in $\Top$.
What do I mean by this?
Let $\cc$ be a category, with objects $\cc_0$ and morphisms $\cc_1$.
Then we have maps $\cc_1\times_{\cc_0}\cc_1\xrightarrow{\text{compose}}\cc_1$ and two maps (source and target) $\cc_1\to \cc_0$, and the identity $\cc_0\to\cc_1$.
You can give the exact same data in any category with pullbacks.

An example is a topological group -- that's a category in $\Top$.
So, if a topological group acts on a space $X$, then we have $XG$, which is a category in $\Top$.
Then $(XG)_0 = X$ as a space, and $(XG)_1 = G\times X$ as a space.

Well, I could go on.
The nerve of a topological category gives me a simplicial space.
In general, you'll have $(N\cc)_n = \cc_1\times_{\cc_0}\cc_1\times\cdots\times_{\cc_0}\cc_1$.
The geometric realization works exactly the same way, so you get a topological space when you realize a simplicial space.

You need to put some mild topological conditions on the group for the theorem we proved above to be true.
Again, we have $\widetilde{G}$ with $G$ acting on it.
Again, we let $EG = B\widetilde{G}G$.
\begin{theorem}
    $EG$ is contractible, and $G$ acts from the right principally (at least if $G$ is a Lie group -- more generally, absolute neighborhood retract)\footnote{If you want to work with the $p$-adics or something, there's a different construction you have to use.}, with orbit projection $EG\to BG$.
\end{theorem}
