\section{Spectral sequences: applications}
pset 5 is complete; it's due next Wednesday.

Leray originally invented his spectral sequences for locally compact things, but $\Omega S^n$ for instance isn't locally compact.
\begin{example}
    What's $\Omega S^1$?
    This is the base of a fibration $\Omega S^1 \to PS^1 \to S^1$, so $\Omega S^1 \simeq \Z$ since we have another fibration $\Z\to \RR\to S^1$.

    For our fibration ($n>1$) $\Omega S^n\to PS^n \to S^n$, we have $S^n$ which has torsion-free homology. Recall the $E^2$-page:
    $$
    E^2_{s,t} = H_s(B;H_t(F)) \simeq H_s(B)\otimes H_t(F)
    $$
    The second isomorphism follows since the homology of the base is torsion-free via universal coefficients.

    Our spectral sequence runs:
    $$E^2_{s,t} = H_s(S^n) \otimes H_t(\Omega S^n) \Rightarrow H_{\ast}(PS^n) = \Z$$
    According to this formula, the $E^2$-page is concentrated in columns $s=0,n$.
    Now, we know that $H_0(\Omega S^n) = \Z$.
    We know that $E^2_{n,0}$ has to be killed.
    It has to die somewhere by being hit by a differential or supporting a nonzero differential.

    There's not very many possibilities for differentials in this spectral sequence.
    There's only one thing that can possibly happen: you have a nonzero $d^n:E^2_{n,0} \to E^n_{0,n-1}$.
    Note that $E^2\simeq E^3 \simeq \cdots \simeq E^n$.
    The differential has to be a monomorphism because if it has some kernel, that's going to be left over!

    We don't know what $E^n_{0,n-1}$ is, yet.
    If this is bigger than $\Z$, then $d^n$ isn't surjective, so that this is the last chance to kill everything in $(0,n-1)$.
    Thus, $d^n$ has to be an epimorphism.
    (If it's not, there'll be cycles that aren't boundaries, and there'll be something left over in $E^{n+1}$ since that stuff'll be left over all the way up to $E^\infty$.)
    We find that $E^n_{0,n-1} \simeq \Z$ and $d^n$ is an isomorphism.
    This $\Z$ is not a big surprise.

    We've now discovered that $H_{n-1}(\Omega S^n) \simeq \Z$.
    That says that I'm not done!
    We still have a $\Z$ in $E^n_{n,n-1}$.
    That also has to die!
    I'm in exactly the same situation as before, so $d^n:E^n_{n,n-1}\to E^n_{0,2(n-1)}$ has to be an isomorphism again.
    In particular, we obtain:
    \begin{equation*}
	H_q(\Omega S^n) \simeq \begin{cases}
	    \Z & \text{if }(n-1)|q\geq 0\\
	    0 & \text{else}
	\end{cases}
    \end{equation*}
\end{example}
That process is an essential part about spectral sequences.
It is a great example of how useful they can be, but misses a few key points.
\begin{remark}
    The loops $\Omega X$ is an associative $H$-space.
    Thus, $H_\ast(\Omega X; R)$ is a graded associative algebra (this is true for any $H$-space).
    We have an adjunction with $\Sigma$ as the left adjoint and $\Omega$ the right adjoint.
    We have the unit $A\to \Omega \Sigma A$.
    Thus, we have a map $\widetilde{H}_\ast(A)\to H_\ast(\Omega \Sigma A)$.
    We do have the universal tensor algebra $\mathrm{Tens}(\widetilde{H}_\ast(A))$ -- it's the free associative algebra on $\widetilde{H}_\ast(A)$, so that it's
    $$
    \bigoplus_{n\geq 0}\widetilde{H}_\ast(A)^{\otimes n}
    $$
    In particular we get a map $\alpha:\mathrm{Tens}(\widetilde{H}_\ast(A))\to H_\ast(\Omega \Sigma A)$.
    \begin{theorem}[Bott-Samelson]
	The map $\alpha$ is an isomorphism if $R$ is a PID and $H_\ast(A)$ is torsion-free.
    \end{theorem}
    \begin{proof}
	See George Whitehead's book.
    \end{proof}
    For instance, if $A = S^{n-1}$ then $\Omega S^n = \Omega \Sigma A$.
    We then have:
    $$
    H_\ast(\Omega S^n) = \mathrm{Tens}(\widetilde{H}_\ast(S^{n-1})) = \langle 1, x, x^2, x^3, \cdots\rangle
    $$
    where $|x| = n-1$.
    It's a mistake to call this polynomial since if $n$ is even, $x$ is an odd class.
\end{remark}
Note that $H_\ast(\Omega S^n)$ acts on this spectral sequence.
Then the $d^r$'s are linear (module homomorphisms).
\begin{remark}
    $\Omega A$ is homotopy equivalent to $\Omega_M A$ where $\Omega_M A$ is a topological monoid.
    I mean it's got a strict unit and is strictly associative (with no homotopy involved).
    Let me describe $\Omega_M A$.
    These are called the Moore loops.
    This is the space $\{(\ell,\omega):\ell\in\RR_{\geq 0}, \omega:[0,\ell]\to A, \omega(0) = \ast = \omega(\ell)\}$.
    It's topologized as a subspace of the product.
    There's an identity class $1\in\Omega_M A$ with $1 = (0,c_\ast)$.
    When I want to add two loops I'll just concatenate (but the lengths get added).
    So, it's strictly associative.
    It's not hard to see that $\Omega A\hookrightarrow \Omega_M A$ is a homotopy equivalence (if the basepoint is nondegenerate).

    Now, I can form the free monoid $\mathrm{FreeMon}(A)$.
    I form a new space whose elements are just formal sequences of elements of $A$, and multiplication is given by juxtaposition (with topology coming from the product topology).
    I'll adjoin a new element called $1$, namely the constant $1 = \ast$.
    There's a universal map $A\to\mathrm{FreeMon}(A)$.
    I have another map $A\to\Omega\Sigma A$ to a monoid.
    By the universality, we get a monoid map $\beta:\mathrm{FreeMon}(A)\to\Omega\Sigma A$.
    \begin{theorem}[I. James]
	This map $\beta$ is a weak equivalence if $A$ is path-connected.
    \end{theorem}
    The free monoid looks very much like the tensor product.
    Let me just say one more thing.
    Let $J(A) = \mathrm{FreeMon}(A)$.
    \begin{theorem}[James again]
	There's a splitting:
	$$
	\Sigma J(A) \simeq_w \Sigma \left(\bigvee_{n\geq 0}A^{\wedge n}\right)
	$$
    \end{theorem}
    If I apply homology to this, then (since homology commutes with suspension):
    $$
    H_\ast(J(A)) \simeq \bigoplus_{n\geq 0} H_\ast(A^{\wedge n})
    $$
    If we're over a PID and $H_\ast(A)$ is torsion-free, then this is just $\bigoplus_{n\geq 0}H_\ast(A)^{\otimes n}$.
    We can get back our computation of $H_\ast(\Omega S^n)$ from this general fact.
\end{remark}
\subsection{Another example}
Suppose $\pi:E\to B$ is a fibration and $\widetilde{H}_s(B) = 0$ for $s<p$ where $p\geq 1$.
Also suppose that $B$ is path-connected.
Pick $\ast\in B$.
Let $F = \pi^{-1}(\ast)$.
Everything is connected so this is fine.
Assume $\widetilde{H}_t(F) = 0$ for $t<q$, where $q\geq 1$.
It's fairly highly connected.
I want to use the Serre spectral sequence.
To get rid of local coefficients, we will assume that $\pi_1(B)$ acts trivially on $H_\ast(F)$.
I have some coefficient ring in mind.

In our spectral sequence, we have:
$$
E^2_{s,t} = H_s(B;H_t(F)) \Rightarrow H_{s+t}(E)
$$
We have $E^2_{0,0}=\Z$, and $E^2_{0,t} = 0$ for $t<q$.
Also $E^2_{s,0} = 0$ for $s<p$.
In particular, $E^2i_{0,q+t} = H_{q+t}(F)$ and $E^2_{p+k,0} = H_{p+k}(B)$.
There are dragons elsewhere.

What can the spectral sequence look like?
The way I've drawn it, $q<p$.
So the first possible differential is $d^{p}:H_p(B) \to H_{p-1}(F)$, and $d^{p+q}:H_{p+1}(B)\to H_{p}(F)$.
We're good, but there are differentials that hit $E^2_{p,q}$ (this is the tooth of the dragon).

For $s<p+q-1$, the only differential is $d^s:E^s_{s,0}\to E^s_{0,s-1}$.
This is called a ``transgression'', although I don't know why.
It's the last possible differential, possibly nonzero.
Thus, the cokernel of $d^s$ is $E^\infty_{0,s-1}$.
And we have $E^\infty_{s,0}\to E^s_{s,0}$.
We have:
$$
0\to E^\infty_{s,0}\to E^s_{s,0} \simeq H_s(B) \xrightarrow{d^s} E^s_{0,s-1}\simeq H_{s-1}(F) \to E^\infty_{0,s-1}\to 0
$$
There's a mysterious relationship that appears in this spectral sequence.

Something else happens.
Let $n<p+q-1$.
We have $F_0 H_n(E) = E^\infty_{0,n}$.
(Recall that $F_s H_n(E) = \img(H_\ast(\pi^{-1}(\mathrm{sk}_s(B)))\to H_\ast(E))$.
That's the convergence statement, because $F_{-1} = 0$.
So, this maps to $H_n(E)$.
There's only one other potentially nonzero filtration in this range of dimensions.
We have a short exact sequence:
$$
0\to F_0 H_n(E) = E^\infty_{0,n} \to H_n(E) \to E^\infty_{n,0} \to 0
$$
You can put our two sexseqs together.
When you splice them together, you get a lexseq:
$$
H_{p+q-1}(F)\to \cdots\to H_n(F)\to H_n(E)\to H_n(B) \xrightarrow{\text{transgression}} H_{n-1}(F) \to H_{n-1}(E)\to \cdots
$$
This is called the \emph{Serre exact sequence}.
In this range of dimensions, homology behaves like homotopy.
