\section{Edge homomorphisms, transgression}
Recall the Serre spectral sequence for a fibration $F\to E\to B$ has $E^2$-page given by
$$
E^2_{s,t} = H_s(B;H_t(F)) \Rightarrow H_{s+t}(E)
$$
In our situation as before, we get a lexseq from the spectral sequence if $B$ is path-connected, $\widetilde{H}_t(F) = 0$ for $t<q$, $\widetilde{H}_s(B) = 0$ for $s<p$, and $\pi_1(B)$ acts trivially on $H_\ast(F)$:
$$
H_{p+q-1}(F)\xrightarrow{\bullet} H_{p+q-1}(E)\to H_{p+q-1}(B)\to H_{p+q-2}(F)\to\cdots
$$
I'll describe the $\bullet$ arrow.
Recall that $H_t(F) = H_0(B;H_t(F)) = E^2_{0,t}$ surjects onto $E^3_{0,t}$, and this surjects onto $E^4_{0,t}$.
(I think here $t > q$.)
Eventually, from $E^{t+1}_{0,t}$ onwards, it stabilizes.
In particular, $E^{t+1}_{0,t} \simeq E^\infty_{0,t} \simeq \gr_0 H_t(E) \simeq F_0 H_t(E)$, which embeds into $H_t(E)$.
\emph{That's the map $\bullet$!}
This is what's known as an ``edge homomorphism'', which relates the edge to the thing you're converging to.

We have a map $i:F\to E$, and I claim that this induces the same map $\bullet$.
We almost saw this in the construction of a sseq for a filtered complex.
Recall that $F_0H_t(C) = \img(H_t(F_0 C) \to H_t(C))$.
For us, filtration zero is the preimage of the zero skeleton, and since $B$ is simply connected, we find that filtration zero is exactly the fiber $F$.
This doesn't exactly prove what we want.
You can think of this in terms of the following diagram:
\begin{equation*}
    \xymatrix{
	F\ar[r]\ar[d] & F\ar[d]\\
	F\ar[r]\ar[d] & E\ar[d]\\
	\ast\ar@{^(->}[r] & B
    }
\end{equation*}
which induces a map of spectral sequences and proves the desired result.


What about the map $H_s(E)\to H_s(B)$?
Let's see.
The group $F_s H_s(E) = H_s(E)$ maps onto $\gr_s H_s(E) \simeq E^\infty_{s,0}$.
Let's assume that $F$ is connected.
Then $H_s(B) = H_s(B;H_0(F)) = E^2_{s,0}$.
Since $E^3 = \ker d^2$, and so on, we have injections $E^\infty_{s,0} \simeq E^{s+1}_{s,0} \to E^s_{s,0} \to \cdots \to E^3_{s,0} \to E^2_{s,0}$.
This is the desired map $H_s(E) \to H_s(B)$ in the Serre exact sequence.
This is another edge homomorphism.
In fact, this is the map induced by $E\to B$, which can be proved by looking at this equally stupid fibration.
\begin{equation*}
    \xymatrix{
	F\ar[r]\ar[d] & \ast\ar[d]\\
	E\ar[r]\ar[d] & B\ar[d]\\
	B\ar[r] & B
    }
\end{equation*}
We're slowly identifying the maps.

What about this boundary map $\partial:H_{p+q-1}(B)\to H_{p+q-2}(F)$ that appears?
This is called a \emph{transgression}.
This is actually the map $d^s$.
Transgressions can be set up in general, but not on the whole of the $E^2$-term, though.
For us, $E^2_{n,0} = H_n(B)$, where we're again assuming that $F$ is connected.
Now, we have $E^3_{n,0} \to E^2_{n,0}$ that's an injection, and so on, so you have an injection $E^n_{n,0}\to H_n(B)$.
But I have $d^n:E^n_{n,0}\to E^n_{0,n-1}$, and $E^n_{0,n-1}$ is a quotient of $E^{n-1}_{0,n-1}$, and so on, so that $E^2_{0,n-1} = H_{n-1}(F)$ is a quotient of $E^n_{0,n-1}$.
We are assuming again that $B$ is connected and $\pi_1(B)$ acts trivially on $H_\ast(F)$.
The transgression is only defined on the subset of $E^2_{n,0}$ that survive to $E^n$.
Thus, this is a ``linear relation''.
The transgression in general is not a function in general!
In our case, it is a function, since everything survives to $E^n_{n,0}$ as nothing maps into $E^2_{n,0}$.
(Through all of this, $n\geq 2$).

We can try to identify what this relation is.
We have a map $H_n(E,F)\xrightarrow{\pi_\ast} H_n(B,\ast)$.
We also have a map $\partial : H_n(E,F) \to H_{n-1}(F)$.
My claim is that:
$$
\img \pi_\ast = \img(E^n_{n,0}\to H_n(B) = E^2_{n,0}),\quad \partial\ker\pi_\ast = \ker(H_{n-1}(F) = E^2_{0,n-1} \to E^n_{0,n-1})
$$
The first statement is that the domains agree, and the second is saying that the indeterminacies agree.
Actually, though this looks technical, it's almost obvious.
\begin{proof}[``Proof''.]
    Let $x\in H_n(B)$.
    First thing is to represent this by a cycle, so that $x = [c]$ with $c\in Z_n(B)$.
    Then I take that cycle and lift it to a chain in the total space.
    Now, that chain won't be a cycle (we saw this for $\eta$, the Hopf fibration); it'll have a boundary.
    Recording this boundary is what the differentials do.
    Saying that this class $x = [c]$ survives to $E^n$ is the same as saying that you can find a lift to a chain $\sigma$ in $E$ with $d\sigma\in S_{n-1}(F)$, and $d^n x$ is represented by $[dc]\in H_{n-1}(F)$.
    That's how we computed the $n$th differential.

    That's what the two statements are saying: I lift something from $H_n(B,\ast)$, but this is well-defined up to $F$, so I get something in $H_n(E,F)$.
    But now I just send it to $H_{n-1}(F)$.
    Thus the two statements follow.
\end{proof}
We now have a geometric viewpoint on edge homomorphisms and transgressions.
\subsection{An example}
Suppose we have $\Omega X\to PX \to X$, where we pick $\ast\in X$, a connected space.
What does the sseq look like here?
How do I want to approach this?

Ah, before I get there -- and I will get there -- here's something I want to say first.
I have this Serre exact sequence:
$$
H_{p+q-1}(F)\xrightarrow{\bullet} H_{p+q-1}(E)\to H_{p+q-1}(B)\to H_{p+q-2}(F)\to\cdots
$$
and we also have the homotopy exact sequence:
$$
\ast\to \pi_{p+q-1}(F)\to \pi_{p+q-1}(E)\to \pi_{p+q-1}(B)\to \pi_{p+q-2}(F)\to \cdots
$$
We have Hurewicz maps $\pi_{p+q-1}(X)\to H_{p+q-1}(X)$, and I claim that I have a map of exact sequences between these two.
The diagram commutes by naturality of Hurewicz; we know this immediately for everything except for $\partial : H_{p+q-1}(B)\to H_{p+q-2}(F)$.
I have a diagram:
\begin{equation*}
    \xymatrix{
	& & H_{p+q-1}(E,F)\ar[dl]\ar[dr] & &\\
	H_{p+q-1}(E) \ar[r]^{\pi_\ast} & H_{p+q-1}(B)\ar[rr]_\partial & & H_{p+q-2}(F)\ar[r] & \cdots\\
	\pi_{p+q-1}(E)\ar[r]_{\pi_\ast}\ar[dr]\ar[u]_{h} & \pi_{p+q-1}(B)\ar[u]^h\ar[rr] & & \pi_{p+q-2}(F)\ar[r]\ar[u]^h & \cdots\\
	& \pi_{p+q-1}(E,F)\ar[uuur]\ar[urr]\ar[u]^\cong_{\pi_\ast} & &
    }
\end{equation*}
We also have:
$$
\pi_n(E,F) = \pi_{n-1}(\mathrm{hofib}(F\to E)) = \pi_{n-1}(\Omega B) = \pi_n(B)
$$
In particular we get the long arrow in the diagram that makes the square commute.

OK, let's go back to our example.
Suppose $\ast \in X$ and $X$ is simply connected.
Let $p\geq 2$ and suppose that $\widetilde{H}_s(X) = 0$ for $s<p$.
By the Serre spectral sequence we know that the homology of $\Omega X$ begins in dimension $p-1$ since $PX\simeq \ast$, so $q = p-1$.
So, $d^{p-1}:H_n(X) \to H_{n-1}(\Omega X)$ is an isomorphism for $2\leq n\leq 2p-2$ (I think).
\begin{remark}
    If we knew $\widetilde{H}_n(\Omega X) = 0$ for $n<p-1$, then the same argument show that $\widetilde{H}_n(X) = 0$ for $n<p$.
\end{remark}
Now I can reveal a proof of our surprise guest, the Hurewicz theorem!
\subsection*{Hurewicz theorem}
What's the Hurewicz theorem say?
\begin{theorem}[Hurewicz, Serre's proof]
    Let $p\geq 1$.
    Suppose $X$ is a pointed space with $\pi_i(X) = 0$ for $i<p$.
    Then $\widetilde{H}_i(X) = 0$ for $i<p$ and $\pi_p(X)^{ab}\to H_p(X)$ is an isomorphism.
\end{theorem}
\begin{proof}
    Our proof will be assuming Poincar\'{e}'s theorem, assuming that $\pi_1(X)^{ab}\xrightarrow{\simeq} H_1(X)$.
    I'm only going to use this when $X$ is a loop space where $\pi_1$ is already abelian.

    Let's prove this by induction.
    Assume it's true for $p-1$.
    Let's use the loop space fibration.
    We know that $\pi_i(\Omega X) = 0$ for $i<p-1$ by our assumptions.
    By induction, $\widetilde{H}_i(\Omega X) = 0$ for $i<p-1$, and $\pi_{p-1}(\Omega X) \xrightarrow{\simeq} H_{p-1}(\Omega X)$.
    We discover, by our comments above, that $\widetilde{H}_i(X) = 0$ for $i<p$, and $\pi_p(X)\xrightarrow{h}H_p(X)$. We have:
    \begin{equation*}
	\xymatrix{
	    \pi_{p-1}(\Omega X)\ar[r]^\simeq & H_{p-1}(\Omega X)\\
	    \pi_p(X)\ar[u]^\simeq \ar[r]_h & H_p(X)\ar[u]^{\simeq}_{\text{transgression}}
	    }
    \end{equation*}
    And the result follows.
\end{proof}
This proof has an enormous advantage, since you can make modifications that modify all primes except for a single prime, or get rational information.
In other words, it's amenable to localizations.
On Monday we'll talk about Serre classes and get information about homotopy groups way beyond conductivity of the space, if you do it right.
