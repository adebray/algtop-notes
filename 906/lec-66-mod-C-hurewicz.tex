\section{Mod $\cC$ Hurewicz, Whitehead, cohomology spectral sequence}
We had $\cC_{fg}$ and $\cC_{tors}$, and
$$
\cC_\cP = \{A|\ell:A\xrightarrow{\simeq} A,\ell\not\in\cP\},\quad \cC_p = \cC_{\{p\}},\quad \cC_{p^\prime} = \cC_{\text{not }p}
$$
Another one is $\cC_{p^\prime}\cap \cC_{tors}$, which consists of torsion groups such that $p$ is an isomorphism on $A$.
There is therefore no $p$-torsion, and it has only prime-to-$p$ torsion.
This is the same thing as saying that $A\otimes\Z_{(p)} = 0$.
\begin{theorem}[Mod $\cC$ Hurewicz]
    Let $X$ be simply connected and $\cC$ a Serre class such that $A,B\in\cC$ implies that $A\otimes B,\Tor_1(A,B)\in \cC$ (this is axiom 2).
    Assume also that if $A\in\cC$, then $H_j(K(A,1)) = H_j(BA)\in\cC$ for all $j>0$.
    (This is valid for all our examples, and is what is called Axiom 3.)

    Let $n\geq 1$.
    Then $\pi_i(X)\in\cC$ for any $1<i<n$ if and only if $H_i(X)\in\cC$ for any $1<i<n$,
    and $\pi_n(X)\to H_n(X)$ is a mod $\cC$ isomorphism.
\end{theorem}
\begin{example}
    For $1<i<n$, the group $\widetilde{H}_i(X)$ is:
    \begin{enumerate}
	\item torsion;
	\item finitely generated;
	\item finite;
	\item $-\otimes \Z_{(p)} = 0$
    \end{enumerate}
    if and only if $\pi_i(X)$ for $1<i<n$.
\end{example}
\begin{proof}
    Look at $\Omega X\to PX \to X$.
    Then $\pi_1 \Omega X\in\cC$.
    Look at Davis+Kirk.
\end{proof}
There's a Whitehead theorem that comes out of this, that I want to state for you.
\begin{theorem}[Mod $\cC$ Whitehead theorem]
    Let $\cC$ be a Serre class satisfying axioms 1, 2, 3, and:
    \begin{enumerate}
	\item[($2^\prime$)] $A\in\cC$ implies that $A\otimes B\in\cC$ for any $B$.
    \end{enumerate}
    This is satisfied for all our examples except $\cC_{fg}$.

    Suppose I have $f:X\to Y$ where $X,Y$ are simply connected.
    Suppose $\pi_2(X)\to \pi_2(Y)$ is onto.
    Let $n\geq 2$.
    Then $\pi_i(X)\to \pi_i(Y)$ is a $\cC$-isomorphism for $2\leq i\leq n$ and is a $\cC$-epimorphism for $i=n$,
    with the same statement for $H_i$.
\end{theorem}
These kind of theorems help us work locally at a prime, and that's super.
You'll see this in the next assignment, which is mostly up on the web.
You'll also see this in calculations which we'll start doing in a day or two.

Change of subject here.
Today I'm going to say a lot of things for which I won't give a proof.
I want to talk about cohomology sseq.
\subsection{Cohomology sseq}
We're building up this powerful tool using spectral sequences.
We saw how powerful the cup product was, and that is what cohomology is good for.
In cohomology, things get turned upside down:
\begin{definition}
    A \emph{decreasing filtration} of an object $A$ is
    $$A\supseteq\cdots\supseteq F^{-1} A\supseteq F^0 A \supseteq F^1 A\supseteq F^2 A\supseteq \cdots\supseteq 0$$
    This is called ``bounded above'' if $F^0 A = A$.
    Write $\gr^s A = F^sA/F^{s+1}A$.
\end{definition}
\begin{example}
    Suppose $X$ is a filtered space.
    So there's an increasing filtration $\emptyset=F_{-1}X\subseteq F_0X\subseteq\cdots$.
    Let $R$ be a commutative ring of coefficients.
    Then I have $S^\ast(X)$, where the differential goes up one degree.
    Define
    $$F^s S^\ast(X) = \ker(S^\ast(X)\to S^\ast(F_{s-1}X))$$
    For instance, $F^0 S^\ast(X) = S^\ast(X)$.
    Thus this is a bounded above decreasing filtration \todo{My computer will run out of juice soon, \TeX this up later!}.
\end{example}
\begin{example}
    Let $X=E\xrightarrow{\pi}B = \text{ CW-complex}$ with $\pi_1(B)$ acting trivially on $H_t(F)$.
    Then $F_s E = \pi^{-1}(\mathrm{sk}_s B)$.
    Thus I get a filtration on $S^\ast(E)$, and
    $$
    F^s H^\ast(X) = \ker(H^\ast(X)\to H^\ast(F_{s-1}X))
    $$
\end{example}
Doing everything the same as before, we get a \emph{cohomology spectral sequence}.
Here are some facts.
\begin{enumerate}
    \item First, you have $E_r^{s,t}$ (note that indices got reversed).
	There's a differential $d_r:E^{s,t}_r \to E_r^{s+r,t-r+1}$, so that the total degree of the differential is $1$.
    \item You discover that
	$$
	E^{s,t}_2 \simeq H^s(B;H^t(F))
	$$
    \item and $E^{s,t}_\infty \simeq \gr^s H^{s+t}(E)$.
    \item 
\end{enumerate}
