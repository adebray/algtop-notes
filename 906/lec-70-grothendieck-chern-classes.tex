\section{Grothendieck's construction of Chern classes}
No class Monday because I'll be out!
We have a few lectures left and I'll use them to talk about characteristic classes, and we'll see some applications.
Maybe we'll have time to give a construction of Steenrod operations as well.
\subsection{Generalities on characteristic classes}
I want to study $G$-bundles, where $G$ is a topological group.
Let's say $A$ is an abelian group and $n\geq 0$ is an integer.
\begin{definition}
    A \emph{characteristic class} for principal $G$-bundles (with values in $H^n(-;A)$) is a natural transformation
    $$
    \Bun_G(X) \xar{c} H^n(X;A)
    $$
    as functor $\Top\to \Ab$.
    By naturality, I mean that if $f:X\to Y$ and some principal $G$-bundle $P\to Y$, I can get $f^\ast P\to X$.
    Naturality means that $c(f^\ast P) = f^\ast c(P)$.
\end{definition}
You're supposed to think: $\Bun_G(X)$ is mysterious, but $H^n(X;A)$ is ``computable''.
At least, we have the Mayer-Vietoris sequence, and we can often compute that cohomology.

Of course, we know that $\Bun_G(X) = [X,BG]$.
And, although I didn't completely give this story, if $X$ is a CW-complex, then $H^n(X;A) = [X,K(A,n)]$.
They're easier to compute because the $K(A,n)$ form an infinite loop space.
Anyway, by Yoneda, these are maps $BG\to K(A,n)$, i.e., an element of $H^n(BG;A)$.

Well, we have an example already: that's the Euler class.
The Euler class took oriented real $n$-plane vector bundles (with a chosen orientation) and produced an $n$-dimensional cohomology class $e:\Vect^{or}_n(X) = \Bun_{SO(n)}(X)\to H^n(X;\Z)$.
What I'm claiming is that if I take $\xi\downarrow X$, then I can pull that back via $f:X \to Y$.
Then $f^\ast\xi$ has a orientation if $\xi$ does, and I'm claiming that $e(f^\ast\xi) = f^\ast e(\xi)$.
But this is kinda obvious from the stuff we did before because everything we did was natural.

The \emph{mod $2$ Euler class} is $e_2:\Vect_n(X) = \Bun_{O(n)}(X) \to H^n(X;\Z/2\Z)$.
Everything has an orientation with respect to $\Z/2\Z$ so this makes sense.
These give you obstructions to nontriviality of the vector bundle.

The Euler class $e$ lives in $H^n(BSO(n);\Z)$; in fact, it's $e(\xi)$, i.e., the Euler class of the universal oriented $n$-plane bundle over $BSO(n)$. A similar statement holds for $e_2\in H^n(BO(n);\Z/2\Z)$.

Something useful about characteristic classes:
if two bundles over $X$ have two different characteristic classes, then they can't be the same.
Often, you use it to distinguish it from the trivial bundle.

I want to show that the Euler class is really fundamental.
For instance, if $n=2$, then $SO(2) = S^1$, and so $BSO(2) = BS^1 = \CP^\infty$.
We know that $H^\ast(\CP^\infty;\Z) = \Z[e]$ -- it's the polynomial algebra on that Euler class.
You can't get much more fundamental than that!
Another example: we know that $O(1) = \Z/2\Z$, and $BO(1) = \RP^\infty$.
We know that $H^\ast(\RP^\infty;\Z/2\Z) = \Z/2\Z[e_2]$ -- it's the polynomial algebra over $\Z/2\Z$ on that mod $2$ Euler class.

I want to now describe Chern classes.
\subsection{Chern classes}
\begin{theorem}[Chern classes]
    There is a unique family of characteristic classes for complex vector bundles that assigns to a complex $n$-plane bundle $\xi$ over $X$ the \emph{$n$th Chern class} $c^{(n)}_k(\xi)\in H^{2k}(X;\Z)$, such that:
    \begin{enumerate}
	\item $c^{(n)}_0(\xi) = 1$.
	\item If $\xi$ is a line bundle, then $c^{(1)}_1(\xi) = -e(\xi)$. (I'll justify this sign in a few minutes.)
	\item The \emph{Whitney sum formula} is true:
	    \begin{equation*}
		c^{(p+q)}_k(\xi\oplus \eta) = \sum_{i+j=k} c^{(p)}_i(\xi)\cup c^{(q)}_j(\eta) \in H^{2k}(X;\Z);
	    \end{equation*}
	    that's the fiberwise direct sum there; here $\xi$ is a $p$-plane bundle and $\eta$ is a $q$-plane vector bundle.
    \end{enumerate}
    Moreover:
    $$H^\ast(BU(n);\Z) \simeq \Z[c_1^{(n)}, \cdots, c^{(n)}_n]$$
    where $c^{(n)}_k = c^{(n)}_k(\xi_n)$, where $\xi_n$ is the universal $n$-plane bundle.
\end{theorem}
This says that all characteristic classes for complex vector bundles are given by polynomials in the Chern classes
because the cohomology of $BU(n)$ gives all the characteristic classes.
It also says that there are no universal algebraic relations among the Chern classes: you can specify them independently.

There were a bunch of questions about computing Chern classes.

How do I interpret the Whitney sum formula in terms of $H^\ast(BU(n))$?
I have $\xi_{p+q}$ over $BU(p+q)$.
Note that $\xi$ with a subscript is the universal bundle.
Ah, but I also have $\xi_p\times \xi_q = pr_1^\ast \xi_p\oplus pr_2^\ast\xi_q$ over $BU(p)\times BU(q)$.
This is a $(p+q)$-plane bundle, and hence is classified by a map $BU(p)\times BU(q) \xar{\mu} BU(p+q)$.
The Whitney sum formula is computing the effect of $\mu$ on cohomology:
$$
\mu^\ast(c^{(n)}_k) = \sum_{i+j = k} c^{(p)}_i \times c^{(q)}_j \in H^{2k}(BU(p)\times BU(q))
$$
where, you'll recall, that $x\times y = pr_1^\ast x \cup pr_2^\ast y$, where $x\in H^\ast(X),y\in H^\ast(Y)$.

Another important comment is stability.
Let $\epsilon$ be the trivial one-dimensional complex vector bundle.
Let $\xi$ be an $n$-dimensional vector bundle.
Then what is $c^{(n+q)}_k(\xi\oplus\epsilon^q)$?
We need to know the Chern classes of the trivial bundles.

Note that the trivial bundle is characterized by the pullback:
\begin{equation*}
    \xymatrix{
	X\times \cc^n = n\epsilon \ar[r]\ar[d] & \cc^n\ar[d]\\
	X\ar[r] & \ast
    }
\end{equation*}
By naturality, we find that $c^{(n)}_k(n\epsilon) = 0$, where $k>0$.
By the Whitney sum formula, we therefore find that
$$
c^{(n+q)}_k(\xi\oplus \epsilon^q) = c^{(n)}_k(\xi)
$$
This is called stability.
It doesn't depend precisely on the vector bundle: so that could simplify the bundle a lot.
For this reason, we'll drop the superscript, and just write $c_k(\xi)$.

The truth is that Chern classes are defined on K-theory.

Let me try to give you Grothendieck's construction of them now.
Here's his idea.
\subsection{(One of) Grothendieck's construction(s)}
Let $\xi$ be an $n$-plane bundle.
Then I can consider $\PP(\xi)$, the projectivization of $\xi$.
This is a vector bundle over $X$, with projection map $\pi:\PP(\xi) \to X$.
An element of the fiber of $\PP(\xi)$ over $x\in X$ is just a line inside $\xi_x$.
The fibers are therefore all isomorphic to $\CP^{n-1}$.

I want to understand $H^\ast(\PP(\xi))$.
And, well, we can now use the Serre spectral sequence, which runs
$$
E_2^{s+t} = H^s(X; H^t(\CP^{n-1})) \Rightarrow H^{s+t}(\PP(\xi))
$$
\begin{remark}
Why didn't I put the underline?
I can look at the universal example of an $n$-plane bundle $\xi_n\to BU(n)$.
Our space $X$ need not be simply connected, but $BU(n)$ is simply connected since $U(n)$ is simply connected.
I can look at the projectivization of the $\xi_n$, and so I can pullback via $f:X\to BU(n)$ to get $\PP(\xi)$.
We actually have $H_\ast(\PP(\xi)_x) \to H_\ast(\PP(\xi_n)_{f(x)})$, which is an isomorphism.
This map is equivariant with respect to the action of the fundamental group $\pi_1 (X) \to \pi_1(BU(n)) = 0$.
\end{remark}
What does this spectral sequence look like?
Because $H^\ast(\CP^{n-1})$ is torsion-free and finitely generated in each dimension, we know that $E_2^{s,t} \simeq H^s(X) \otimes H^t(\CP^{n-1})$.

I claim that the spectral sequence collapses at $E_2$, i.e., that $E_2 \simeq E_\infty$, i.e., no differentials.
This is like the proof of the Leray-Hirsch theorem.
We know that it's generated as an algebra by things on the fiber and things on the base.
Thus it suffices to check that things on the fiber survive to $\infty$.
We know that $E_2^{0,2t} = \Z\langle x^t\rangle$ and $E_2^{0,2t+1} = 0$, where $x = e(\lambda)$ where $\lambda\downarrow\CP^{n-1}$.
Why does the Euler class $x$ survive?
It only could support a $d_3$, but why does it survive?

Can you come up with a two dimensional cohomology class in $\PP(\xi)$ that restricts to the Euler class over $\CP^{n-1}$?
Well, $\lambda$ itself is the restriction of a line bundle over $\CP^\infty$.
There's a tautologous line bundle $\lambda_\xi \downarrow \PP(\xi)$ which is given by the tautologous line bundle on each fiber.
In other words, $E(\lambda_\xi)\ni(x\in \ell\subseteq\xi_x)$.
Thus, $x = e(\lambda_\xi)|_\text{fiber}$.
This class $x$ therefore survives, and I have an explicit lift to the cohomology of the total space!

So, what we discover is that (Leray-Hirsch kicks in)
$$
H^\ast(\PP(\xi)) = H^\ast(X)\langle 1, e(\lambda_\xi), e(\lambda_\xi)^2, \cdots, e(\lambda_\xi)^{n-1}\rangle
$$
Let's just write $e = e(\lambda_\xi)$.
We know some of the algebra structure, but I don't know that $e^n$ is!
We know that $e^n$ can't be in filtration zero anymore: it must lie in lower filtration.
Whatever it is, it's going to be a linear combination of those coefficients.
I'll choose to write that as a monic polynomial:
$$
e^n + c_1e^{n-1} + \cdots + c_{n-1} e + c_n = 0
$$
Where the $c_k \in H^{2k}(X)$.
Those are the Chern classes, and they are unique!

For instance, $e+c_1 = 0$ if $n=1$. See?

\begin{question}
    Why do we know that they are the Chern classes?
    My margin is too narrow to provide a proof.
\end{question}
We'll prove this on Wednesday.
