\section{$H^\ast(BU(n))$, splitting principle}
I wanted to share some ideas about the question on the first nonzero homology group of an Eilenberg-MacLane space.
One idea was to use this extension $0\to \Z\xar{p^k}\Z\to \Z/p^k\to 0$.
This gives rise to a fiber sequence $K(\Z,n)\to K(\Z,n)\to K(\Z/p^k,n)\to 0$.
You could maybe use the Serre spectral sequence here?
Another idea that came up was that this is embedded in a long sequence of fibrations, so that you have $K(\Z,n)\to K(\Z,n)\to K(\Z/p^k, n) \to K(\Z,n+1)\to\cdots$.

I'm really excited today -- it's more characteristic classes!
Last week, I gave you Grothendieck's construction of Chern classes:
if you have a complex $n$-plane bundle $\xi\downarrow X$, I described $\PP(\xi)$, which has a canonical tautologous line bundle $\lambda_\xi\downarrow \PP(\xi)$.
This has an Euler class $e=e(\lambda_\xi)\in H^2(\PP(\xi))$, and this satisfies the following unique relation:
$$
e^n+c_1e^{n-1} + \cdots + c_n = 0
$$
Then $c_k\in H^{2k}(X)$ are called the Chern classes.

This doesn't give you much insight as to how to compute them.
I also claimed that these Chern classes generate the cohomology of $BU$ as a polynomial algebra.
I'll talk about that today.
\subsection{$H^\ast(BU(n))$ -- how to compute it}
Over $BU(n)$, I have the universal principal $U(n)$-bundle $EU(n)\to BU(n)$.
And then, given any left action of $U(n)$ on some space, I can form the associated fiber bundle.
Suppose we consider the $U(n)$-action on $\cc^n$; then $EU(n)\times_{U(n)}\cc^n\to BU(n)$ is the universal line bundle $\xi_n$.
I could also consider the action on $S^{2n-1}\subseteq \cc^n$.
Then I'd get $S(\xi)$, the unit sphere bundle; the fiber of $EU(n)\times_{U(n)}S^{2n-1}\to BU(n)$ is $S^{2n-1}$.
Now, we also know that $S^{2n-1} = U(n)/(1\times U(n-1))$.
Thus, I can write
$$EU(n)\times_{U(n)}S^{2n-1} = EU(n)\times_{U(n)} (U(n)/U(n-1)) = EU(n)/U(n-1) = BU(n-1)$$
which shows that $BU(n-1)$ is the sphere bundle of the tautologous line bundle over $BU(n)$.
Thus we get a fiber bundle:
$$
S^{2n-1}\to BU(n-1)\to BU(n)
$$
We can use this as an inductive tool, via the Serre spectral sequence.
We've already talked about the Serre spectral sequence for a spherical fibration, and we found that all the information was contained in the Gysin sequence.

Recall the Gysin sequence of a $S^{2n-1}$-fiber bundle.
Suppose $S^{2n-1}\to E\xar{\pi} B$ is our fiber bundle, with $B$ oriented.
How does the Gysin sequence go?
$$
\cdots\to H^{q-1}(E) \xar{\pi_\ast} H^{q-2n}(B) \xar{e\cdot} H^q(B) \xar{\pi^\ast} H^q(E) \xar{\pi_\ast} \cdots
$$
Note that there are no twisted coefficients, because in our scenario, $\pi_1 BU(n) = \pi_0 U(n) = 0$.

Let's make two inductive assumptions:
\begin{enumerate}
    \item $H^\mathrm{odd}(E) = 0$ -- this works whether we start the induction at $n=1$ or $n=2$ (respectively, you have $\ast = BU(0)$ and $\CP^\infty = BU(1)$).

	This implies that if $q$ is even, then $\pi_\ast = 0$, which implies that $e|_{H^\mathrm{even}(B)}$ is monic.
	If $q$ is odd, then $e\cdot H^{q-2n}(B) = H^q(B)$.
	In the latter case, what if $q=1$?
	Then $H^{q-2n}(B) = 0$, so by induction on $q$, you find that $H^\mathrm{odd}(B) = 0$.
	If $q$ is even, then $H^{q-2n+1}(B) = 0$, and so we get a sexseq
	$$
	0\to H^\ast(B) \xar{e\cdot} H^\ast(B) \to H^\ast(E) \to 0
	$$
	I.e., the cohomology of $E$ is the cohomology of $B$ quotiented by the ideal generated by $e$.
	We've also discovered that $e\in H^{2n}(B)$ is a nonzero divisor.

	\begin{example}
	    $n=1$, $B=\CP^\infty$, $E\simeq \ast$.
	    I have $e\in H^2(\CP^\infty)$.
	    What these deductions mean is that $H^\ast(\CP^\infty) \simeq \Z[e]$.
	    This is good.
	\end{example}
    \item The cohomology $H^\ast(E)$ is polynomial.
	Well, now I have this surjection $H^\ast(B) \xar{\pi^\ast} H^\ast(E)$.
	Since $H^\ast(E)$ is polynomial, we can lift the generators to something in $H^\ast(B)$.
	That'll give a splitting $s:H^\ast(E) \to H^\ast(B)$.
	I also have this extra element in the cohomology of $B$, namely, the Euler class.
	Thus, we get a map $H^\ast(E)[e] \xar{\overline{s}} H^\ast(B)$.
	This is going to be an isomorphism -- why?

	This is a standard argument.
	Filter both sides by powers of $e$, i.e., take the $e$-adic filtration.
	On the associated graded of $H^\ast(E)[e]$, we get the cohomology of $E$ everytime.
	On $H^\ast(B)$, we get the same thing, because $H^\ast(B)/eH^\ast(B) \simeq H^\ast(E)$.
	Thus the associated graded $\gr^\ast(\overline{s})$ is an isomorphism.
	In this particular case (but not in general), you conclude that $\overline{s}$ is an isomorphism, because in any single dimension, the filtration is finite.
	Thus you can use the five lemma over and over again to see that you have an isomorphism on each filtered piece, which tells us that $\overline{s}$ is an isomorphism.
\end{enumerate}
Following that induction, we get that
$$
H^\ast(BU(n-1)) = \Z[c_1,\cdots,c_{n-1}]
$$
Note that I haven't checked the axioms for the $c_i$ yet.
Notice that there's a map $\pi^\ast:H^\ast(BU(n)) \to H^\ast(BU(n))$ that's an isomorphism in dimensions $<2n$.
Thus these classes have \emph{unique} lifts (so $s$ is unique) to $H^\ast(BU(n))$.
We therefore get:
\begin{theorem}
    There exist classes $c_i\in H^{2i}(BU(n))$ for $1\leq i\leq n$ such that:
    \begin{itemize}
	\item $H^\ast(BU(n)) \xar{\pi_\ast} H^\ast(BU(n-1))$ sends
	    $$
	    c_i \mapsto \begin{cases}
		c_i & i<n\\
		0 & i=n
	    \end{cases}
	    $$
	\item $c_n := (-1)^n e\in H^{2n}(BU(n))$, to make the theorem work out properly -- this is still a generator.
    \end{itemize}
    And
    $$
    \boxed{H^\ast(BU(n)) \simeq \Z[c_1,\cdots,c_n]}
    $$
\end{theorem}
It says that complex vector bundles are pretty complicated.
There might be mod $p$ implications among them, but these are the characteristic classes.
I want to go a little further.
\subsection{Whitney sum formula}
The tool is the splitting principle.
We've already proved it.
\begin{theorem}
    Let $\xi^n\downarrow X$.
    Then there exists a space $\Fl(\xi) \xar{\pi} X$ such that:
    \begin{enumerate}
	\item $\pi^\ast \xi = \lambda_1\oplus\cdots\lambda_n$, where the $\lambda_i$ are line bundles.
	\item the map $\pi^\ast: H^\ast(X) \to H^\ast(\Fl(\xi))$ is monic.
    \end{enumerate}
\end{theorem}
We already have this -- that's what the projective bundle does for you!
We've computed that
$$
H^\ast(\PP(\xi)) = H^\ast(X)\langle 1,e,\cdots,e^{n-1}\rangle
$$
Iterate.
And we're done.

That's no fun -- let's be a little more explicit about $\Fl(\xi)$, and why I call it $\Fl$.
Remember the \emph{frame bundle} of $\xi$, which is $Fr(\xi)$, and element of which is a map $\cc^n\xar{\text{linear, inner-product-preserving}}E(\xi)$.
There are various constructions I can make with this; for instance, $E(\xi) = Fr(\xi)\times_{U(n)}\cc^n$, where $\xi$ is the principal $U(n)$-bundle.
In addition, $\PP(\xi) = Fr(\xi)\times_{U(n)} U(n)/(1\times U(n-1))$.

The \emph{flag bundle} is
$$\Fl(\xi) = Fr(\xi)\times_{U(n)} U(n)/(U(1)\times\cdots\times U(1))$$
This thing $U(1)\times\cdots\times U(1)$ is called $T^n$; it's the maximal torus in $U(n)$.
In the universal case, this frame bundle is $EU(n)$, and so $\Fl(\xi_n)$ is $BT^n\to BU(n)$, whose fiber is $U(n)/T^n$.
The spaces $EU(n)$ and $\PP(\xi)$ are obtained by splitting off line bundles, and so we find that $\pi:\Fl(\xi)\to X$ is a monomorphism in cohomology.
That's very cool, because we now find that $H^\ast(BU(n)) \hookrightarrow H^\ast(BT^n)$.
The cohomology of $BT^n$ is extremely simple -- it's the cohomology of a product of $\CP^\infty$s.
In fact, $H^\ast(BT^n) \simeq \Z[t_1,\cdots,t_n]$, where $|t_k| = 2$.
These $t_i$s are the Euler classes of $\lambda_i$, under $\pi_i:BT^n\to\CP^\infty$.
