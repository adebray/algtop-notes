\section{Stiefel-Whitney classes, immersions, cobordisms}
\begin{theorem}
    There exist a unique family of characteristic classes $w_i:\Vect_n(X) \to H^n(X;\FF_2)$ such that for $0\leq i$ and $i>n$, we have $w_i=0$, and:
    \begin{enumerate}
	\item $w_0 = 1$
	\item $w_1(\lambda) = e(\lambda)$.
	\item Whitney sum:
	    \begin{equation*}
		w_k(\xi\oplus\eta) = \sum_{i+j=k} w_i(\xi)\cup w_j(\eta)
	    \end{equation*}
    \end{enumerate}
    Moreover:
    $$
    H^\ast(BO(n);\FF_2) = \FF_2[w_1,\cdots,w_n]
    $$
    where $w_n = e_2$, and $H^\ast(BO(n-1);\FF_2) = \FF_2[w_1,\cdots,w_{n-1}]$ with $w_i\mapsto w_i$ for $i<n$ and $0$ for $i=n$
\end{theorem}
\begin{remark}
    You can express the Whitney sum formula simply by defining the \emph{total Steifel-Whitney class} $1 + w_1 + w_2 + \cdots=:w$ (this is an inhomogeneous cohomology class).
    Then the Whitney sum formula is just
    $$
    w(\xi\oplus\eta) = w(\xi)\cdot w(\eta)
    $$
    That's kind of how you prove it, actually, with the splitting principle.
\end{remark}
Here, Grothendieck's definition works.
The splitting principle also holds, so $H^\ast(BO(n))\hookrightarrow H^\ast(B(\Z/2\Z)^n)$.
We can't do what we did before, when considering $S^{n-1} \to EO(n)\times_{O(n)} O(n)/O(n-1)\to BO(n)$, because the $n-1$ can be even or odd; there's no restriction for it to be even like it was before.
We still have a Gysin sequence, though:
$$
\cdots\to H^{q-n}(BO(n))\xar{e\cdot} H^q(BO(n)) \xar{\pi^\ast} H^q(BO(n-1)) \to H^{q-n+1}(BO(n))\to\cdots
$$
We want that $e$ is a nonzero divisor.
But we have a splitting principle!

We know that $H^\ast(BO(n)) \hookrightarrow H^\ast((\RP^\infty)^n)$, and $e_2=w_n\mapsto e_2(\lambda_1\oplus\cdots\oplus\lambda_n) = t_1\cdots t_n$, which \emph{is} a nonzero divisor, as $H^\ast((\RP^\infty)^n)$ is an integral domain.
\todo{this was not written well, because I came late.}
\subsection{Application I: immersions}
I'm going to think about immersions into Euclidean space.
That means that I have some manifold $M^n$, that is some closed smooth manifold.
An immersion is a smooth map $f:M^n\looparrowright \RR^{n+k}$ (that's how it's denoted) such that $(\tau_{M^n})_x \hookrightarrow (\tau_{\RR^{n+k}})_{f(x)}$ for $x\in M$.
So you can have crossings, but no cusps.
It's a big problem in topology to try to eliminate the crossings.

By the way, if you haven't seen it, there's a beautiful immersion $\RP^2 \looparrowright \RR^3$ that's known as \emph{Boys' surface}.

That's a question; when can I immerse a manifold into an Euclidean space?
Assume I had an immersion $i:M^n\looparrowright \RR^{n+k}$.
Then I have $f:\tau_M \to i^\ast \tau_{\RR^{n+k}}$.
The right hand thing is a trivial bundle over $M$, and the property of being an immersion says that $f$ is an embedding.
Thus, $\tau_M$ has a $k$-dimensional complement, called $\xi$.
This $\tau_M\oplus\xi = (n+k)\epsilon$.
When I apply the total Steifel-Whitney class, we have (since there's no higher Steifel-Whitney class of a trivial bundle):
$$
w(\tau)w(\xi) = 1
$$
Which means that
$$
w(\xi) = w(\tau)^{-1}
$$
\begin{example}
    Let $M = \RP^n\looparrowright \RR^{n+k}$.
    I picked this since we know what $\tau_{\RP^n}$ is.
    In particular, we know that
    $$
    \tau_{\RP^n}\oplus\epsilon\simeq (n+1)\lambda^\ast \simeq (n+1)\lambda
    $$
    where $\lambda\downarrow \RP^n$ is the canonical line bundle.
    Now, I can compute the total Steifel-Whitney class of this.
    By the way, because of that Whitney sum formula, we get that $w(\xi\oplus k\epsilon) = w(\xi)$ -- these are \emph{stable} characteristic classes.
    So, I have:
    $$
    w(\tau_{\RP^n}) = w(\tau_{\RP^n}\oplus\eta) = w((n+1)\lambda) = w(\lambda)^{n+1}
    $$
    What's $w(\lambda)$?
    It only has a $w_1$, and thus $w(\lambda) = x$ if we write $H^\ast(\RP^n) = \FF_2[x]/(x^{n+1})$; everything is with $\FF_2$-coefficients.
    Thus, we have
    $$
    w(\tau_{\RP^n}) = (1+x)^{n+1} = \sum^n_{i=0}\binom{n+1}{i}x^i
    $$
    and thus it follows that
    $$
    w_i(\tau_{\RP^n}) = \binom{n+1}{i}x^i
    $$
    If we're talking about immersions, we should talk about the SW class of the complement of the tangent bundle, and it's:
    $$
    w(\xi) = (1+x)^{-n-1}
    $$
    We could write down a formula, but the most interesting case is what happens when $n=2^s$ for some integer $s$.
    Then we'd have
    $$
    w(\xi) = (1+x)^{-1-2^s} = (1+x)^{-1}(1+x)^{-2^s} = (1+x)^{-1}(1+x^{2^s})^{-1} = (1+x+x^2+\cdots)(1+x^{2^s}+\cdots)
    $$
    Because taking powers of $2$ is linear in char. $2$.
    Since we're in $\FF_2[x]/x^{2^s+1}$, I don't care about higher terms.
    So
    $$
    w(\xi) = 1+x+x^2+\cdots+x^{2^s-1}+2x^s = 1+x+x^2+\cdots+x^{2^s-1}
    $$
    This means that $k = \dim\xi \geq 2^s-1$, as $x^{2^s-1}\neq 0$.
    The conclusion is:
    \begin{theorem}
	$\RP^{2^s}\not \looparrowright \RR^{2\cdot 2^{s}-2}$.
    \end{theorem}
    This is sharp\footnote{This means that I can't reduce the dimension of the Euclidean space into which I'm embedding.}, because of the following theorem:
    \begin{theorem}[Whitney]
	Any smooth compact closed manifold $M^n \looparrowright \RR^{2n-1}$.
	This result is \emph{not} sharp.
    \end{theorem}
    The sharp theorem, though, is this:
    \begin{theorem}[Brown, Peterson, Cohen, in 1980?]
	A closed compact smooth $n$-manifold $M^n \looparrowright \RR^{2n-\alpha(n)}$, where $\alpha(n)$ is the number of $1$s in the dyadic expansion of $n$.
	This is sharp, since if $n=\sum 2^{d_i}$ for the dyadic expansion, then $M = \prod_i \RP^{2^{d_i}} \not \looparrowright \RR^{2n-\alpha(n)-1}$, using Steifel-Whitney classes.
    \end{theorem}
\end{example}
There was a period of time in the late '60s, etc. when a lot of effort was put in to stronger and stronger immersion results.
\subsection{Cobordism, characteristic numbers}
If I have an $n$-manifold (smooth closed compact), one thing that's always true is that it embeds in $\RR^{n+k}$ for some $k>>0$.
So, you have a normal bundle: $\tau_M^n\oplus \nu_M^k = (n+k)\epsilon$.
A piece of differential topology tells you that if $k$ is large, then $\nu_M\oplus ?\epsilon$ is independent of the bundle, where $?$ is some number.

It makes sense to talk about the Steifel-Whitney class for $w\in H^\ast(BO) = \FF_2[w_1,w_2,\cdots]$, where $BO = \varinjlim BO(n)$.
That's where we should be talking about things that are well-defined but only up to some trivial bundle.
I can consider $w(\nu_M)$ -- that makes sense.
The $w$ is to remind you that this is related to the Steifel-Whitney class.
I can now consider $\langle w(\nu_M),[M]\rangle$.
This is called the characteristic number.
We saw an example of this already, for instance in the Euler characteristic (this is slightly different since I'm using the normal bundle here).
Note that $[M]$ exists since we're working with coeffs in $\FF_2$.

That's a useful thing you can do.
This relates to the notion of a cobordism.

Let me say that two $n$-manifolds $M, N$ are \emph{(co)bordant} if there's an $(n+1)$-dimensional manifold $W^{n+1}$ with boundary such that $\partial W\simeq M\sqcup N$.
What if $n=0$, for example?
We know that the manifold $\ast\sqcup \ast$ is \emph{not} cobordant to $\ast$, but it is cobordant to the empty set.
But $\ast\sqcup\ast\sqcup\ast$ is cobordant to $\ast$.
Any manifold is cobordant to itself, since $\partial(M\times I) = M\sqcup M$.
Cobordism is an equivalence relation.

Note that all my manifolds are compact and smooth.
But $M$ and $N$ are closed.

Let us define
$$
\Omega^O_n = \{\text{cobordism classes of $n$-manifolds}\}
$$
We computed that $\Omega^O_n = \FF_2$.
I wrote this as a group, because I can add cobordism classes, by taking disjoint unions.
Every element is its own inverse.
In fact, when I put this together, I get a graded \emph{cobordism ring}, denoted $\Omega^O_\ast$.

Let me show you something here.
This is due to Rene Thom.
He made the following observation, that's pretty cool.
Suppose my manifold is embedded into Euclidean space.
Then suppose $M$ is nullbordant via some $W$.
When you set it up like that, you find that $\nu_W|_{M} = \nu_M$; there's no change in dimension.

Look what I can do.
I can consider $\langle w(\nu_M),[M]\rangle = \langle w(\nu_W)|_{M},[M]\rangle$.
On the other hand, $H_n(M) \leftarrow H_{n+1}(W,M)$ given by $[W,M]\mapsto[M]$, where that's a relative fundamental class.
Thus $\langle w(\nu_M),[M]\rangle = \langle w(\nu_M),\partial[W,M]\rangle = \langle \delta w(\nu_M),[W,M]\rangle$.
But, let's see, we have an exact sequence $H^n(W)\xar{i^\ast} H^n(M) \xar{\delta} H^{n+1}(W,M)$.
Since $w(\nu_M)$ is $i^\ast$ of something, it follows that $\delta w(\nu_M) = 0$.
Thus, we find that:
\begin{prop}
    Characteristic numbers are cobordism invariants.
    That is to say, they give a map $\Omega^O_n \to \Hom(H^n(BO),\FF_2)\simeq H_n(BO)$, by UCT.
\end{prop}
Thus Stiefel-Whitney numbers tell all.
\begin{theorem}[Thom, 1954]
    This map of graded rings is an embedding, i.e., $\Omega^O_\ast \hookrightarrow H_\ast(BO)$.
    In other words, if $w(M^n) = w(N^n)$ for all $w\in H^n(BO)$, then $M^n \sim N^n$.
\end{theorem}
The way that he did this was by expressing $\Omega^O_\ast = \pi_\ast(\text{space})$, and he showed that that space i the product of mod $2$ Eilenberg-MacLane spaces.
Along the way, he also gave a computation:
$$
\Omega^O_\ast = \FF_2[x_i:i\neq 2^s-1] = \FF_2[x_2,x_4,x_5,x_6,x_8,\cdots]
$$
(It's a nice exercise to see that every $1$-manifold is nullbordant.)
On Wednesday we'll talk about Pontryagin classes and the signature theorem.
